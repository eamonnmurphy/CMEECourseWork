% Options for packages loaded elsewhere
\PassOptionsToPackage{unicode}{hyperref}
\PassOptionsToPackage{hyphens}{url}
%
\documentclass[
]{article}
\title{Practical session 2}
\author{Sandra Alvarez-Carretero}
\date{3/1/2022}

\usepackage{amsmath,amssymb}
\usepackage{lmodern}
\usepackage{iftex}
\ifPDFTeX
  \usepackage[T1]{fontenc}
  \usepackage[utf8]{inputenc}
  \usepackage{textcomp} % provide euro and other symbols
\else % if luatex or xetex
  \usepackage{unicode-math}
  \defaultfontfeatures{Scale=MatchLowercase}
  \defaultfontfeatures[\rmfamily]{Ligatures=TeX,Scale=1}
\fi
% Use upquote if available, for straight quotes in verbatim environments
\IfFileExists{upquote.sty}{\usepackage{upquote}}{}
\IfFileExists{microtype.sty}{% use microtype if available
  \usepackage[]{microtype}
  \UseMicrotypeSet[protrusion]{basicmath} % disable protrusion for tt fonts
}{}
\makeatletter
\@ifundefined{KOMAClassName}{% if non-KOMA class
  \IfFileExists{parskip.sty}{%
    \usepackage{parskip}
  }{% else
    \setlength{\parindent}{0pt}
    \setlength{\parskip}{6pt plus 2pt minus 1pt}}
}{% if KOMA class
  \KOMAoptions{parskip=half}}
\makeatother
\usepackage{xcolor}
\IfFileExists{xurl.sty}{\usepackage{xurl}}{} % add URL line breaks if available
\IfFileExists{bookmark.sty}{\usepackage{bookmark}}{\usepackage{hyperref}}
\hypersetup{
  pdftitle={Practical session 2},
  pdfauthor={Sandra Alvarez-Carretero},
  hidelinks,
  pdfcreator={LaTeX via pandoc}}
\urlstyle{same} % disable monospaced font for URLs
\usepackage[margin=1in]{geometry}
\usepackage{color}
\usepackage{fancyvrb}
\newcommand{\VerbBar}{|}
\newcommand{\VERB}{\Verb[commandchars=\\\{\}]}
\DefineVerbatimEnvironment{Highlighting}{Verbatim}{commandchars=\\\{\}}
% Add ',fontsize=\small' for more characters per line
\usepackage{framed}
\definecolor{shadecolor}{RGB}{248,248,248}
\newenvironment{Shaded}{\begin{snugshade}}{\end{snugshade}}
\newcommand{\AlertTok}[1]{\textcolor[rgb]{0.94,0.16,0.16}{#1}}
\newcommand{\AnnotationTok}[1]{\textcolor[rgb]{0.56,0.35,0.01}{\textbf{\textit{#1}}}}
\newcommand{\AttributeTok}[1]{\textcolor[rgb]{0.77,0.63,0.00}{#1}}
\newcommand{\BaseNTok}[1]{\textcolor[rgb]{0.00,0.00,0.81}{#1}}
\newcommand{\BuiltInTok}[1]{#1}
\newcommand{\CharTok}[1]{\textcolor[rgb]{0.31,0.60,0.02}{#1}}
\newcommand{\CommentTok}[1]{\textcolor[rgb]{0.56,0.35,0.01}{\textit{#1}}}
\newcommand{\CommentVarTok}[1]{\textcolor[rgb]{0.56,0.35,0.01}{\textbf{\textit{#1}}}}
\newcommand{\ConstantTok}[1]{\textcolor[rgb]{0.00,0.00,0.00}{#1}}
\newcommand{\ControlFlowTok}[1]{\textcolor[rgb]{0.13,0.29,0.53}{\textbf{#1}}}
\newcommand{\DataTypeTok}[1]{\textcolor[rgb]{0.13,0.29,0.53}{#1}}
\newcommand{\DecValTok}[1]{\textcolor[rgb]{0.00,0.00,0.81}{#1}}
\newcommand{\DocumentationTok}[1]{\textcolor[rgb]{0.56,0.35,0.01}{\textbf{\textit{#1}}}}
\newcommand{\ErrorTok}[1]{\textcolor[rgb]{0.64,0.00,0.00}{\textbf{#1}}}
\newcommand{\ExtensionTok}[1]{#1}
\newcommand{\FloatTok}[1]{\textcolor[rgb]{0.00,0.00,0.81}{#1}}
\newcommand{\FunctionTok}[1]{\textcolor[rgb]{0.00,0.00,0.00}{#1}}
\newcommand{\ImportTok}[1]{#1}
\newcommand{\InformationTok}[1]{\textcolor[rgb]{0.56,0.35,0.01}{\textbf{\textit{#1}}}}
\newcommand{\KeywordTok}[1]{\textcolor[rgb]{0.13,0.29,0.53}{\textbf{#1}}}
\newcommand{\NormalTok}[1]{#1}
\newcommand{\OperatorTok}[1]{\textcolor[rgb]{0.81,0.36,0.00}{\textbf{#1}}}
\newcommand{\OtherTok}[1]{\textcolor[rgb]{0.56,0.35,0.01}{#1}}
\newcommand{\PreprocessorTok}[1]{\textcolor[rgb]{0.56,0.35,0.01}{\textit{#1}}}
\newcommand{\RegionMarkerTok}[1]{#1}
\newcommand{\SpecialCharTok}[1]{\textcolor[rgb]{0.00,0.00,0.00}{#1}}
\newcommand{\SpecialStringTok}[1]{\textcolor[rgb]{0.31,0.60,0.02}{#1}}
\newcommand{\StringTok}[1]{\textcolor[rgb]{0.31,0.60,0.02}{#1}}
\newcommand{\VariableTok}[1]{\textcolor[rgb]{0.00,0.00,0.00}{#1}}
\newcommand{\VerbatimStringTok}[1]{\textcolor[rgb]{0.31,0.60,0.02}{#1}}
\newcommand{\WarningTok}[1]{\textcolor[rgb]{0.56,0.35,0.01}{\textbf{\textit{#1}}}}
\usepackage{graphicx}
\makeatletter
\def\maxwidth{\ifdim\Gin@nat@width>\linewidth\linewidth\else\Gin@nat@width\fi}
\def\maxheight{\ifdim\Gin@nat@height>\textheight\textheight\else\Gin@nat@height\fi}
\makeatother
% Scale images if necessary, so that they will not overflow the page
% margins by default, and it is still possible to overwrite the defaults
% using explicit options in \includegraphics[width, height, ...]{}
\setkeys{Gin}{width=\maxwidth,height=\maxheight,keepaspectratio}
% Set default figure placement to htbp
\makeatletter
\def\fps@figure{htbp}
\makeatother
\setlength{\emergencystretch}{3em} % prevent overfull lines
\providecommand{\tightlist}{%
  \setlength{\itemsep}{0pt}\setlength{\parskip}{0pt}}
\setcounter{secnumdepth}{-\maxdimen} % remove section numbering
\ifLuaTeX
  \usepackage{selnolig}  % disable illegal ligatures
\fi

\begin{document}
\maketitle

\hypertarget{setting-your-working-environment}{%
\section{Setting your working
environment}\label{setting-your-working-environment}}

First, we will clean and set our working environment. It is important
that, whenever you start working on a project, you clean your working
environment to avoid issues with objects generated in previous
sessions/projects. If you want to do this using the command line, you
can do the following:

\begin{Shaded}
\begin{Highlighting}[]
\CommentTok{\# Clean environment }
\FunctionTok{rm}\NormalTok{( }\AttributeTok{list =} \FunctionTok{ls}\NormalTok{( ) )}

\CommentTok{\# Set working directory with package \textasciigrave{}rstudioapi\textasciigrave{}:}
\CommentTok{\#}
\CommentTok{\# 1. Load the package \textasciigrave{}rstudioapi\textasciigrave{}. If you do not have }
\CommentTok{\#    it installed, then uncomment and run the}
\CommentTok{\#    command below}
\CommentTok{\# install.packages( "rstudioapi" )}
\FunctionTok{library}\NormalTok{( rstudioapi ) }
\CommentTok{\# 2. Get the path to current open R script}
\NormalTok{path\_to\_file }\OtherTok{\textless{}{-}} \FunctionTok{getActiveDocumentContext}\NormalTok{()}\SpecialCharTok{$}\NormalTok{path}
\NormalTok{wd           }\OtherTok{\textless{}{-}} \FunctionTok{paste}\NormalTok{( }\FunctionTok{dirname}\NormalTok{( path\_to\_file ), }\StringTok{"/"}\NormalTok{, }\AttributeTok{sep =} \StringTok{""}\NormalTok{ )}
\FunctionTok{setwd}\NormalTok{( wd )}
\end{Highlighting}
\end{Shaded}

Now, any data that you generate when you run the commands in this R
script will be saved in the directory you have defined above (unless you
specify otherwise when saving the data!).

\begin{center}\rule{0.5\linewidth}{0.5pt}\end{center}

\hypertarget{prior-likelihood-unnormalised-posterior-and-posterior}{%
\section{Prior, likelihood, unnormalised posterior, and
posterior}\label{prior-likelihood-unnormalised-posterior-and-posterior}}

\emph{\textbf{SCENARIO - part 1}} Your data will be now an alignment 12s
RNA gene sequences from human and orangutan. You have checked this
alignment and you know that there are a total of \(n = 948\)
nucleotides. When you compare the two sequences, you can count the
differences to estimate how closely related they are. In total, you
observe \(x = 90\) differences.

Now, we want to be able to estimate the molecular distance, \(d\),
between the two sequences, which will help us understand how closely
related these two species are. There are different models of nucleotide
substitution that we could use, but we will focus on the somplest one
based on the assumption that human and orangutan are quite similar: we
will use the Jukes and Cantor (JC69) model.

\emph{\textbf{QUESTION 1}} Write down the posterior distribution in a
Bayesian framework (i.e., prior, likelihood, marginal likelihood). Find
out what your parameter of interest is and what data you have.

\emph{\textbf{SCENARIO - part 2}} The functions that we will be using
for our prior and likelihood are the following:

\begin{Shaded}
\begin{Highlighting}[]
\CommentTok{\# Likelihood function under the JC69 model}
\NormalTok{L\_d }\OtherTok{\textless{}{-}} \ControlFlowTok{function}\NormalTok{(d, x, n)\{}
  \FunctionTok{return}\NormalTok{( (}\DecValTok{3}\SpecialCharTok{/}\DecValTok{4} \SpecialCharTok{{-}} \DecValTok{3}\SpecialCharTok{*}\FunctionTok{exp}\NormalTok{(}\SpecialCharTok{{-}}\DecValTok{4}\SpecialCharTok{*}\NormalTok{d}\SpecialCharTok{/}\DecValTok{3}\NormalTok{)}\SpecialCharTok{/}\DecValTok{4}\NormalTok{)}\SpecialCharTok{\^{}}\NormalTok{x }\SpecialCharTok{*}\NormalTok{ (}\DecValTok{1}\SpecialCharTok{/}\DecValTok{4} \SpecialCharTok{+} \DecValTok{3}\SpecialCharTok{*}\FunctionTok{exp}\NormalTok{(}\SpecialCharTok{{-}}\DecValTok{4}\SpecialCharTok{*}\NormalTok{d}\SpecialCharTok{/}\DecValTok{3}\NormalTok{)}\SpecialCharTok{/}\DecValTok{4}\NormalTok{)}\SpecialCharTok{\^{}}\NormalTok{(n }\SpecialCharTok{{-}}\NormalTok{ x) )}
\NormalTok{\}}

\CommentTok{\# Prior function on my parameter of interest.}
\CommentTok{\# We will assume that an exponential distribution }
\CommentTok{\# fits best and that the mean will be 0.2}
\NormalTok{prior\_d }\OtherTok{\textless{}{-}} \ControlFlowTok{function}\NormalTok{(d, mu)\{}
  \FunctionTok{return}\NormalTok{( }\FunctionTok{exp}\NormalTok{(}\SpecialCharTok{{-}}\NormalTok{d}\SpecialCharTok{/}\NormalTok{mu) }\SpecialCharTok{/}\NormalTok{ mu )}
\NormalTok{\}}
\end{Highlighting}
\end{Shaded}

\emph{\textbf{QUESTION 1}} Can you define the unnormalised posterior?

\begin{Shaded}
\begin{Highlighting}[]
\NormalTok{unnorm\_post }\OtherTok{\textless{}{-}} \ControlFlowTok{function}\NormalTok{(d, x, n, mu)}
\NormalTok{\{}
  \FunctionTok{return}\NormalTok{(}\FunctionTok{prior\_d}\NormalTok{(}\AttributeTok{d =}\NormalTok{ d, }\AttributeTok{mu =}\NormalTok{ mu) }\SpecialCharTok{*} \FunctionTok{L\_d}\NormalTok{(}\AttributeTok{d =}\NormalTok{ d, }\AttributeTok{x =}\NormalTok{ x, }\AttributeTok{n =}\NormalTok{ n))}
\NormalTok{\}}
\end{Highlighting}
\end{Shaded}

\emph{\textbf{QUESTION 2}} Use the \texttt{integrate} function to find
the normalising constant \(C\) and then write the function for the
posterior distribution.

\begin{Shaded}
\begin{Highlighting}[]
\NormalTok{marg\_likelihood }\OtherTok{\textless{}{-}} \FunctionTok{integrate}\NormalTok{(}\AttributeTok{f =} \FunctionTok{Vectorize}\NormalTok{(unnorm\_post), }\AttributeTok{lower =} \DecValTok{0}\NormalTok{, }\AttributeTok{upper =} \ConstantTok{Inf}\NormalTok{,}
                             \AttributeTok{x =} \DecValTok{90}\NormalTok{, }\AttributeTok{n =} \DecValTok{948}\NormalTok{, }\AttributeTok{mu =} \FloatTok{0.2}\NormalTok{, }\AttributeTok{abs.tol =} \DecValTok{0}\NormalTok{)}
\NormalTok{C }\OtherTok{\textless{}{-}} \DecValTok{1} \SpecialCharTok{/}\NormalTok{ marg\_likelihood}\SpecialCharTok{$}\NormalTok{value}

\NormalTok{posterior }\OtherTok{\textless{}{-}} \ControlFlowTok{function}\NormalTok{(d, x, n, mu)}
\NormalTok{\{}
  \FunctionTok{return}\NormalTok{(C }\SpecialCharTok{*} \FunctionTok{unnorm\_post}\NormalTok{(}\AttributeTok{d =}\NormalTok{ d, }\AttributeTok{x =}\NormalTok{ x, }\AttributeTok{n =}\NormalTok{ n, }\AttributeTok{mu =}\NormalTok{ mu))}
\NormalTok{\}}
\end{Highlighting}
\end{Shaded}

\emph{\textbf{QUESTION 3}} Plot the posterior and the prior together.
What can you tell about your Bayesian inference?

\begin{Shaded}
\begin{Highlighting}[]
\NormalTok{d }\OtherTok{\textless{}{-}} \FunctionTok{seq}\NormalTok{(}\DecValTok{0}\NormalTok{, }\DecValTok{1}\NormalTok{, }\AttributeTok{by =} \FloatTok{0.01}\NormalTok{)}
\NormalTok{prior }\OtherTok{\textless{}{-}} \FunctionTok{prior\_d}\NormalTok{(d, }\AttributeTok{mu =} \FloatTok{0.2}\NormalTok{)}
\NormalTok{post }\OtherTok{\textless{}{-}} \FunctionTok{posterior}\NormalTok{(d, }\AttributeTok{x =} \DecValTok{90}\NormalTok{, }\AttributeTok{n =} \DecValTok{948}\NormalTok{, }\AttributeTok{mu =} \FloatTok{0.2}\NormalTok{)}

\FunctionTok{plot.new}\NormalTok{()}
\FunctionTok{lines}\NormalTok{(d, prior, }\AttributeTok{type =} \StringTok{"l"}\NormalTok{)}
\FunctionTok{lines}\NormalTok{(d, post, }\AttributeTok{col =} \StringTok{"red"}\NormalTok{)}
\end{Highlighting}
\end{Shaded}

\includegraphics{220301_Practical_session02_00_ex_files/figure-latex/unnamed-chunk-5-1.pdf}

\emph{\textbf{QUESTION 4 (quite hard)}} \emph{HINT: Use the function
\texttt{integrate}} What is the probability that your parameter of
interest, \(d\), is larger than 0.2?

\begin{Shaded}
\begin{Highlighting}[]
\NormalTok{prob\_d }\OtherTok{\textless{}{-}} \FunctionTok{integrate}\NormalTok{(}\AttributeTok{f =}\NormalTok{ posterior, }\AttributeTok{lower =} \FloatTok{0.2}\NormalTok{, }\AttributeTok{upper =} \DecValTok{1}\NormalTok{, }\AttributeTok{x =} \DecValTok{90}\NormalTok{, }\AttributeTok{n =} \DecValTok{948}\NormalTok{, }\AttributeTok{mu =} \FloatTok{0.2}\NormalTok{)}
\FunctionTok{print}\NormalTok{(prob\_d}\SpecialCharTok{$}\NormalTok{value)}
\end{Highlighting}
\end{Shaded}

\begin{verbatim}
## [1] 9.974834e-13
\end{verbatim}

\end{document}
