% Options for packages loaded elsewhere
\PassOptionsToPackage{unicode}{hyperref}
\PassOptionsToPackage{hyphens}{url}
%
\documentclass[
]{article}
\title{Understanding the MCMC algorithm}
\author{Sandra Alvarez-Carretero and Mario dos Reis}
\date{11/30/2021}

\usepackage{amsmath,amssymb}
\usepackage{lmodern}
\usepackage{iftex}
\ifPDFTeX
  \usepackage[T1]{fontenc}
  \usepackage[utf8]{inputenc}
  \usepackage{textcomp} % provide euro and other symbols
\else % if luatex or xetex
  \usepackage{unicode-math}
  \defaultfontfeatures{Scale=MatchLowercase}
  \defaultfontfeatures[\rmfamily]{Ligatures=TeX,Scale=1}
\fi
% Use upquote if available, for straight quotes in verbatim environments
\IfFileExists{upquote.sty}{\usepackage{upquote}}{}
\IfFileExists{microtype.sty}{% use microtype if available
  \usepackage[]{microtype}
  \UseMicrotypeSet[protrusion]{basicmath} % disable protrusion for tt fonts
}{}
\makeatletter
\@ifundefined{KOMAClassName}{% if non-KOMA class
  \IfFileExists{parskip.sty}{%
    \usepackage{parskip}
  }{% else
    \setlength{\parindent}{0pt}
    \setlength{\parskip}{6pt plus 2pt minus 1pt}}
}{% if KOMA class
  \KOMAoptions{parskip=half}}
\makeatother
\usepackage{xcolor}
\IfFileExists{xurl.sty}{\usepackage{xurl}}{} % add URL line breaks if available
\IfFileExists{bookmark.sty}{\usepackage{bookmark}}{\usepackage{hyperref}}
\hypersetup{
  pdftitle={Understanding the MCMC algorithm},
  pdfauthor={Sandra Alvarez-Carretero and Mario dos Reis},
  hidelinks,
  pdfcreator={LaTeX via pandoc}}
\urlstyle{same} % disable monospaced font for URLs
\usepackage[margin=1in]{geometry}
\usepackage{color}
\usepackage{fancyvrb}
\newcommand{\VerbBar}{|}
\newcommand{\VERB}{\Verb[commandchars=\\\{\}]}
\DefineVerbatimEnvironment{Highlighting}{Verbatim}{commandchars=\\\{\}}
% Add ',fontsize=\small' for more characters per line
\usepackage{framed}
\definecolor{shadecolor}{RGB}{248,248,248}
\newenvironment{Shaded}{\begin{snugshade}}{\end{snugshade}}
\newcommand{\AlertTok}[1]{\textcolor[rgb]{0.94,0.16,0.16}{#1}}
\newcommand{\AnnotationTok}[1]{\textcolor[rgb]{0.56,0.35,0.01}{\textbf{\textit{#1}}}}
\newcommand{\AttributeTok}[1]{\textcolor[rgb]{0.77,0.63,0.00}{#1}}
\newcommand{\BaseNTok}[1]{\textcolor[rgb]{0.00,0.00,0.81}{#1}}
\newcommand{\BuiltInTok}[1]{#1}
\newcommand{\CharTok}[1]{\textcolor[rgb]{0.31,0.60,0.02}{#1}}
\newcommand{\CommentTok}[1]{\textcolor[rgb]{0.56,0.35,0.01}{\textit{#1}}}
\newcommand{\CommentVarTok}[1]{\textcolor[rgb]{0.56,0.35,0.01}{\textbf{\textit{#1}}}}
\newcommand{\ConstantTok}[1]{\textcolor[rgb]{0.00,0.00,0.00}{#1}}
\newcommand{\ControlFlowTok}[1]{\textcolor[rgb]{0.13,0.29,0.53}{\textbf{#1}}}
\newcommand{\DataTypeTok}[1]{\textcolor[rgb]{0.13,0.29,0.53}{#1}}
\newcommand{\DecValTok}[1]{\textcolor[rgb]{0.00,0.00,0.81}{#1}}
\newcommand{\DocumentationTok}[1]{\textcolor[rgb]{0.56,0.35,0.01}{\textbf{\textit{#1}}}}
\newcommand{\ErrorTok}[1]{\textcolor[rgb]{0.64,0.00,0.00}{\textbf{#1}}}
\newcommand{\ExtensionTok}[1]{#1}
\newcommand{\FloatTok}[1]{\textcolor[rgb]{0.00,0.00,0.81}{#1}}
\newcommand{\FunctionTok}[1]{\textcolor[rgb]{0.00,0.00,0.00}{#1}}
\newcommand{\ImportTok}[1]{#1}
\newcommand{\InformationTok}[1]{\textcolor[rgb]{0.56,0.35,0.01}{\textbf{\textit{#1}}}}
\newcommand{\KeywordTok}[1]{\textcolor[rgb]{0.13,0.29,0.53}{\textbf{#1}}}
\newcommand{\NormalTok}[1]{#1}
\newcommand{\OperatorTok}[1]{\textcolor[rgb]{0.81,0.36,0.00}{\textbf{#1}}}
\newcommand{\OtherTok}[1]{\textcolor[rgb]{0.56,0.35,0.01}{#1}}
\newcommand{\PreprocessorTok}[1]{\textcolor[rgb]{0.56,0.35,0.01}{\textit{#1}}}
\newcommand{\RegionMarkerTok}[1]{#1}
\newcommand{\SpecialCharTok}[1]{\textcolor[rgb]{0.00,0.00,0.00}{#1}}
\newcommand{\SpecialStringTok}[1]{\textcolor[rgb]{0.31,0.60,0.02}{#1}}
\newcommand{\StringTok}[1]{\textcolor[rgb]{0.31,0.60,0.02}{#1}}
\newcommand{\VariableTok}[1]{\textcolor[rgb]{0.00,0.00,0.00}{#1}}
\newcommand{\VerbatimStringTok}[1]{\textcolor[rgb]{0.31,0.60,0.02}{#1}}
\newcommand{\WarningTok}[1]{\textcolor[rgb]{0.56,0.35,0.01}{\textbf{\textit{#1}}}}
\usepackage{longtable,booktabs,array}
\usepackage{calc} % for calculating minipage widths
% Correct order of tables after \paragraph or \subparagraph
\usepackage{etoolbox}
\makeatletter
\patchcmd\longtable{\par}{\if@noskipsec\mbox{}\fi\par}{}{}
\makeatother
% Allow footnotes in longtable head/foot
\IfFileExists{footnotehyper.sty}{\usepackage{footnotehyper}}{\usepackage{footnote}}
\makesavenoteenv{longtable}
\usepackage{graphicx}
\makeatletter
\def\maxwidth{\ifdim\Gin@nat@width>\linewidth\linewidth\else\Gin@nat@width\fi}
\def\maxheight{\ifdim\Gin@nat@height>\textheight\textheight\else\Gin@nat@height\fi}
\makeatother
% Scale images if necessary, so that they will not overflow the page
% margins by default, and it is still possible to overwrite the defaults
% using explicit options in \includegraphics[width, height, ...]{}
\setkeys{Gin}{width=\maxwidth,height=\maxheight,keepaspectratio}
% Set default figure placement to htbp
\makeatletter
\def\fps@figure{htbp}
\makeatother
\setlength{\emergencystretch}{3em} % prevent overfull lines
\providecommand{\tightlist}{%
  \setlength{\itemsep}{0pt}\setlength{\parskip}{0pt}}
\setcounter{secnumdepth}{-\maxdimen} % remove section numbering
\ifLuaTeX
  \usepackage{selnolig}  % disable illegal ligatures
\fi

\begin{document}
\maketitle

\hypertarget{part-1}{%
\section{PART 1}\label{part-1}}

\hypertarget{introduction}{%
\subsection{Introduction}\label{introduction}}

The data that we will be using for this practical session is the 12S
rRNA alignment of human and orangutan, which consists of 948 base pairs
and 90 differences (i.e., 84 transitions and 6 transversions):

\begin{quote}
\textbf{Table 1}. Numbers and frequencies (in parantheses) of sites for
the 16 site configurations (patterns) in human and orangutan
mitochondrial 12s rRNA genes. This table is based on Table 1.3, page 7
in \href{http://abacus.gene.ucl.ac.uk/MESA/}{Yang (2014)}.
\end{quote}

\begin{longtable}[]{@{}llllll@{}}
\toprule
Orangutan (below) ~Human (right) & T & C & A & G & Sum (\(\pi_{i}\)) \\
\midrule
\endhead
T & 179 (0.188819) & 23 (0.024262) & 1 (0.001055) & 0 (0) & 0.2141 \\
C & 30 (0.03164646) & 219 (0.231013) & 2 (0.002110) & 0 (0) & 0.2648 \\
A & 2 (0.002110) & 1 (0.001055) & 291 (0.306962) & 10 (0.010549) &
0.3207 \\
G & 0 (0) & 0 (0) & 21 (0.022152) & 169 (0.178270) & 0.2004 \\
Sum(\(\pi_{j}\) & 0.226 & 0.2563 & 0.3323 & 0.1888 & 1 \\
\bottomrule
\end{longtable}

\begin{quote}
\begin{quote}
\emph{Note}: Genbank accesion numbers for the human and orangutan
sequences are \texttt{D38112} and \texttt{NC\_001646}, respectively
(\href{https://pubmed.ncbi.nlm.nih.gov/7530363/}{Horai et al.~(1995)}).
There are 954 sites in the alignment, but six sites involve alignment
gaps and are removed, leaving 948 sites in each sequence. The average
base frequencies in the two sequences are 0.2184 (T), 0.2605 (C), 0.3265
(A), and 0.1946 (G).
\end{quote}
\end{quote}

We are going to use the \href{https://cran.r-project.org/}{R programming
language} to load, parse, and analyse the data. You can also run all the
commands we will go through in this tutorial from the graphical
interface
\href{https://www.rstudio.com/products/rstudio/download/}{RStudio}. If
you are unfamiliar with the installation of both these software, you can
follow a step-by-step tutorial
\href{https://github.com/sabifo4/RnBash/tree/master/R_installation}{here}.
This tutorial has a detailed explanation of each task that you are going
to carry out. Note that you can also find all the code shown and
explained here in \href{Practical_1.R}{this R script}.

\hypertarget{analysing-the-data}{%
\subsection{Analysing the data}\label{analysing-the-data}}

First, we will define the variables for our data set: the length of the
alignment, the total number of transitions (i.e., substitutions between
the two pyrimidines, \texttt{T\ \textless{}-\textgreater{}\ C)}), and
the total number of transversions (i.e., substitutions between the two
purines, \texttt{A\ \textless{}-\textgreater{}\ G}):

\begin{Shaded}
\begin{Highlighting}[]
\CommentTok{\# Length of alignment in bp}
\NormalTok{n  }\OtherTok{\textless{}{-}} \DecValTok{948}
\CommentTok{\# Total number of transitions (23+30+10+21)}
\NormalTok{ns }\OtherTok{\textless{}{-}} \DecValTok{84}
\CommentTok{\# Total number of transversions (1+0+2+0+2+1+0+0)}
\NormalTok{nv }\OtherTok{\textless{}{-}} \DecValTok{6}
\end{Highlighting}
\end{Shaded}

Then, we need to write a function that returns us the log-likelihood
given the distance between two sequences (\(d\)) and the
transition/transversion rate ratio (\(\kappa=\alpha/\beta\)), which is
written as \(f(D|d,k)\). This function might change depending on the
nucleotide substitution model that is to be used. In this practical
session, we will be using Kimura's 1980 (K80) nucleotide substitution
model (see page 8, \href{http://abacus.gene.ucl.ac.uk/MESA/}{Yang
(2014)}), which accounts for the fact that transitions often occur at
higher rates than transversions. Therefore, the parameters to consider
are the following:

\begin{itemize}
\tightlist
\item
  Distance, \(d\).\\
\item
  Kappa, \(\kappa\).\\
\item
  Alignment length, \(n\). In this example, \(n=948\).\\
\item
  Number of transitions, \(ns\). In this example, \(ns=84\).\\
\item
  Number of transversions, \(nv\). In this example, \(nv=6\).
\end{itemize}

\begin{Shaded}
\begin{Highlighting}[]
\CommentTok{\# Define log{-}likelihood function, f(D|d,k).}
\CommentTok{\# This uses Kimura\textquotesingle{}s (1980) substitution model. See p.8 in Yang (2014).}
\CommentTok{\#}
\CommentTok{\# Arguments:}
\CommentTok{\#}
\CommentTok{\#   d  Numeric, value for the distance.}
\CommentTok{\#   k  Numeric, value for parameter kappa.}
\CommentTok{\#   n  Numeric, length of the alignment. Default: 948.}
\CommentTok{\#   ns Numeric, total number of transitions. Default: 84.}
\CommentTok{\#   nv Numeric, total number of transcersions. Default: 6.}
\NormalTok{k80.lnL }\OtherTok{\textless{}{-}} \ControlFlowTok{function}\NormalTok{( d, k, }\AttributeTok{n =} \DecValTok{948}\NormalTok{, }\AttributeTok{ns =} \DecValTok{84}\NormalTok{, }\AttributeTok{nv =} \DecValTok{6}\NormalTok{ ) \{}

  \CommentTok{\# Define probabilities}
\NormalTok{  p0 }\OtherTok{\textless{}{-}}\NormalTok{ .}\DecValTok{25} \SpecialCharTok{+}\NormalTok{ .}\DecValTok{25} \SpecialCharTok{*} \FunctionTok{exp}\NormalTok{( }\SpecialCharTok{{-}}\DecValTok{4}\SpecialCharTok{*}\NormalTok{d}\SpecialCharTok{/}\NormalTok{( k}\SpecialCharTok{+}\DecValTok{2}\NormalTok{ ) ) }\SpecialCharTok{+}\NormalTok{ .}\DecValTok{5} \SpecialCharTok{*} \FunctionTok{exp}\NormalTok{( }\SpecialCharTok{{-}}\DecValTok{2}\SpecialCharTok{*}\NormalTok{d}\SpecialCharTok{*}\NormalTok{( k}\SpecialCharTok{+}\DecValTok{1}\NormalTok{ )}\SpecialCharTok{/}\NormalTok{( k}\SpecialCharTok{+}\DecValTok{2}\NormalTok{ ) )}
\NormalTok{  p1 }\OtherTok{\textless{}{-}}\NormalTok{ .}\DecValTok{25} \SpecialCharTok{+}\NormalTok{ .}\DecValTok{25} \SpecialCharTok{*} \FunctionTok{exp}\NormalTok{( }\SpecialCharTok{{-}}\DecValTok{4}\SpecialCharTok{*}\NormalTok{d}\SpecialCharTok{/}\NormalTok{( k}\SpecialCharTok{+}\DecValTok{2}\NormalTok{ ) ) }\SpecialCharTok{{-}}\NormalTok{ .}\DecValTok{5} \SpecialCharTok{*} \FunctionTok{exp}\NormalTok{( }\SpecialCharTok{{-}}\DecValTok{2}\SpecialCharTok{*}\NormalTok{d}\SpecialCharTok{*}\NormalTok{( k}\SpecialCharTok{+}\DecValTok{1}\NormalTok{ )}\SpecialCharTok{/}\NormalTok{( k}\SpecialCharTok{+}\DecValTok{2}\NormalTok{ ) )}
\NormalTok{  p2 }\OtherTok{\textless{}{-}}\NormalTok{ .}\DecValTok{25} \SpecialCharTok{{-}}\NormalTok{ .}\DecValTok{25} \SpecialCharTok{*} \FunctionTok{exp}\NormalTok{( }\SpecialCharTok{{-}}\DecValTok{4}\SpecialCharTok{*}\NormalTok{d}\SpecialCharTok{/}\NormalTok{( k}\SpecialCharTok{+}\DecValTok{2}\NormalTok{ ) )}

  \CommentTok{\# Return log{-}likelihood}
  \FunctionTok{return}\NormalTok{ ( ( n }\SpecialCharTok{{-}}\NormalTok{ ns }\SpecialCharTok{{-}}\NormalTok{ nv ) }\SpecialCharTok{*} \FunctionTok{log}\NormalTok{( p0}\SpecialCharTok{/}\DecValTok{4}\NormalTok{ ) }\SpecialCharTok{+}
\NormalTok{            ns }\SpecialCharTok{*} \FunctionTok{log}\NormalTok{( p1}\SpecialCharTok{/}\DecValTok{4}\NormalTok{ ) }\SpecialCharTok{+}\NormalTok{ nv }\SpecialCharTok{*} \FunctionTok{log}\NormalTok{( p2}\SpecialCharTok{/}\DecValTok{4}\NormalTok{ ) )}

\NormalTok{\}}
\end{Highlighting}
\end{Shaded}

As we have defined our log-likelihood function, we might want to
visually see how this distribution looks like if I am to ``plug'' in
this function different values for my parameters of interest (\(d\) and
\(\kappa\)) as well as the values that relate to my data (i.e., \(n\),
\(ns\), \(nv\); which relate to my alignment).

For this example, we are going to select 100 values for \(\kappa\) that
range from 0 to 100 and 100 values for \(d\) that range from 0 to 0.3.
Then, we will create a data frame in which we will include all possible
combinations of the values selected for \(d\) and \(\kappa\). In that
way, we will be able to extract the paired values of \(d\) and
\(\kappa\) (i.e., each row in the data frame will correspond to one
combination of the values paired for \(d\) and \(\kappa\)) and use them
for subsequent analyses:

\begin{Shaded}
\begin{Highlighting}[]
\CommentTok{\# Number of values that we want to collect for }
\CommentTok{\# each parameter of interest}
\NormalTok{dim }\OtherTok{\textless{}{-}} \DecValTok{100}
\CommentTok{\# Vector of \textasciigrave{}d\textasciigrave{} values}
\NormalTok{d.v }\OtherTok{\textless{}{-}} \FunctionTok{seq}\NormalTok{( }\AttributeTok{from =} \DecValTok{0}\NormalTok{, }\AttributeTok{to =} \FloatTok{0.3}\NormalTok{, }\AttributeTok{len =}\NormalTok{ dim )}
\CommentTok{\# Vector of \textasciigrave{}k\textasciigrave{} values}
\NormalTok{k.v }\OtherTok{\textless{}{-}} \FunctionTok{seq}\NormalTok{( }\AttributeTok{from =} \DecValTok{0}\NormalTok{, }\AttributeTok{to =} \DecValTok{100}\NormalTok{, }\AttributeTok{len =}\NormalTok{ dim )}
\CommentTok{\# Define grid where we will save all possible }
\CommentTok{\# combinations with the 100 values of \textasciigrave{}d\textasciigrave{} and }
\CommentTok{\# the 100 values of \textasciigrave{}k\textasciigrave{}}
\NormalTok{dk  }\OtherTok{\textless{}{-}} \FunctionTok{expand.grid}\NormalTok{( }\AttributeTok{d =}\NormalTok{ d.v, }\AttributeTok{k =}\NormalTok{ k.v )}
\end{Highlighting}
\end{Shaded}

The resulting data frame object has 10,000 rows and 2 columns: we have
recorded a total of 10,000 paired values of \(d\) and \(\kappa\) that
have resulted from combining the 100 values of that were proposed for
each of these parameters of interest.

Now, we can use this object to extract in the correct order the
``paired'' values of \(\kappa\) and \(d\) so we can ``plug'' them in the
log-likelihood function and compute the corresponding log-likelihood
values for each combination of \(d\) and \(\kappa\).

\begin{Shaded}
\begin{Highlighting}[]
\CommentTok{\# Compute likelihood surface, f(D|d,k)}
\NormalTok{lnL }\OtherTok{\textless{}{-}} \FunctionTok{matrix}\NormalTok{( }\FunctionTok{k80.lnL}\NormalTok{( }\AttributeTok{d =}\NormalTok{ dk}\SpecialCharTok{$}\NormalTok{d, }\AttributeTok{k =}\NormalTok{ dk}\SpecialCharTok{$}\NormalTok{k ), }\AttributeTok{ncol =}\NormalTok{ dim )}
\end{Highlighting}
\end{Shaded}

Once we have saved the output log-likelihood values in the matrix above,
we will need to scale the likelihood to be 1 at the maximum so there are
no numerical issues in subsequent analyses:

\begin{Shaded}
\begin{Highlighting}[]
\CommentTok{\# For numerical reasons, we scale the likelihood to be 1}
\CommentTok{\# at the maximum, i.e., we subtract max(lnL)}
\NormalTok{L }\OtherTok{\textless{}{-}} \FunctionTok{exp}\NormalTok{( lnL }\SpecialCharTok{{-}} \FunctionTok{max}\NormalTok{( lnL ) )}
\end{Highlighting}
\end{Shaded}

At this moment, we have now computed the log-likelihood values for each
pair of \(d\) and \(\kappa\) values provided to the \texttt{K80.lnL()}
function!

In addition, if we had some prior information on these two parameters,
we could also use it to build the prior distributions for \(d\) and
\(\kappa\). Imagine that you contact an expert on the field, and they
tell you that the best distribution that you could use to represent the
information about both \(d\) and \(\kappa\) is the Gamma distribution.
They also tell you that \(\alpha\) parameter for both Gamma
distributions should be \(2\) (parameter that the R function
\texttt{dgamma} has labelled as \texttt{shape}) but that the Gamma prior
on \(d\) should have \(\beta=20\) and the Gamma prior on \(\kappa\)
should have \(\beta=0.1\) (the \(\beta\) values is passed to argument
\texttt{rate} in the \texttt{dgamma} function).

\begin{Shaded}
\begin{Highlighting}[]
\CommentTok{\# Compute prior surface, f(D)f(k)}
\NormalTok{Pri }\OtherTok{\textless{}{-}} \FunctionTok{matrix}\NormalTok{( }\FunctionTok{dgamma}\NormalTok{( }\AttributeTok{x =}\NormalTok{ dk}\SpecialCharTok{$}\NormalTok{d, }\AttributeTok{shape =} \DecValTok{2}\NormalTok{, }\AttributeTok{rate =} \DecValTok{20}\NormalTok{ ) }\SpecialCharTok{*}
               \FunctionTok{dgamma}\NormalTok{( }\AttributeTok{x =}\NormalTok{ dk}\SpecialCharTok{$}\NormalTok{k, }\AttributeTok{shape =} \DecValTok{2}\NormalTok{, }\AttributeTok{rate =}\NormalTok{ .}\DecValTok{1}\NormalTok{ ),}
               \AttributeTok{ncol =}\NormalTok{ dim )}
\end{Highlighting}
\end{Shaded}

Given that now we have our likelihood and our prior distributions, we
might also want to compute the unnormalised posterior now:

\begin{Shaded}
\begin{Highlighting}[]
\CommentTok{\# Compute unnormalised posterior surface, f(d)f(k)f(D|d,k)}
\NormalTok{Pos }\OtherTok{\textless{}{-}}\NormalTok{ Pri }\SpecialCharTok{*}\NormalTok{ L}
\end{Highlighting}
\end{Shaded}

Once we have computed the three surfaces, it is time to plot them!

\begin{Shaded}
\begin{Highlighting}[]
\CommentTok{\# Plot prior, likelihood, and unnormalised posterior surfaces.}
\CommentTok{\# We want one row and three columns.}
\FunctionTok{par}\NormalTok{( }\AttributeTok{mfrow =} \FunctionTok{c}\NormalTok{( }\DecValTok{1}\NormalTok{, }\DecValTok{3}\NormalTok{ ) )}
\CommentTok{\# Prior surface. Note that the \textasciigrave{}contour()\textasciigrave{} function creates a contour plot.}
\FunctionTok{image}\NormalTok{( }\AttributeTok{x =}\NormalTok{ d.v, }\AttributeTok{y =}\NormalTok{ k.v, }\AttributeTok{z =} \SpecialCharTok{{-}}\NormalTok{Pri, }\AttributeTok{las =} \DecValTok{1}\NormalTok{, }\AttributeTok{col =} \FunctionTok{heat.colors}\NormalTok{( }\DecValTok{50}\NormalTok{ ),}
       \AttributeTok{main =} \StringTok{"Prior"}\NormalTok{, }\AttributeTok{xlab =} \StringTok{"distance, d"}\NormalTok{,}
       \AttributeTok{ylab =} \StringTok{"kappa, k"}\NormalTok{, }\AttributeTok{cex.main =} \FloatTok{2.0}\NormalTok{,}
       \AttributeTok{cex.lab =} \FloatTok{1.5}\NormalTok{, }\AttributeTok{cex.axis =} \FloatTok{1.5}\NormalTok{ )}
\FunctionTok{contour}\NormalTok{( }\AttributeTok{x =}\NormalTok{ d.v, }\AttributeTok{y =}\NormalTok{ k.v, }\AttributeTok{z =}\NormalTok{ Pri, }\AttributeTok{nlev=}\DecValTok{10}\NormalTok{, }\AttributeTok{drawlab =} \ConstantTok{FALSE}\NormalTok{, }\AttributeTok{add =} \ConstantTok{TRUE}\NormalTok{ )}
\CommentTok{\# Likelihood surface + contour plot.}
\FunctionTok{image}\NormalTok{( }\AttributeTok{x =}\NormalTok{ d.v, }\AttributeTok{y =}\NormalTok{ k.v, }\AttributeTok{z =} \SpecialCharTok{{-}}\NormalTok{L, }\AttributeTok{las =} \DecValTok{1}\NormalTok{, }\AttributeTok{col =} \FunctionTok{heat.colors}\NormalTok{( }\DecValTok{50}\NormalTok{ ),}
       \AttributeTok{main =} \StringTok{"Likelihood"}\NormalTok{, }\AttributeTok{xlab =} \StringTok{"distance, d"}\NormalTok{,}
       \AttributeTok{ylab =} \StringTok{"kappa, k"}\NormalTok{, }\AttributeTok{cex.main =} \FloatTok{2.0}\NormalTok{,}
       \AttributeTok{cex.lab =} \FloatTok{1.5}\NormalTok{, }\AttributeTok{cex.axis =} \FloatTok{1.5}\NormalTok{ )}
\FunctionTok{contour}\NormalTok{( }\AttributeTok{x =}\NormalTok{ d.v, }\AttributeTok{y =}\NormalTok{ k.v, }\AttributeTok{z =}\NormalTok{ L, }\AttributeTok{nlev =} \DecValTok{10}\NormalTok{,}
         \AttributeTok{drawlab =} \ConstantTok{FALSE}\NormalTok{, }\AttributeTok{add =} \ConstantTok{TRUE}\NormalTok{)}
\CommentTok{\# Unnormalised posterior surface + contour plot.}
\FunctionTok{image}\NormalTok{( }\AttributeTok{x =}\NormalTok{ d.v, }\AttributeTok{y =}\NormalTok{ k.v, }\AttributeTok{z =} \SpecialCharTok{{-}}\NormalTok{Pos, }\AttributeTok{las =} \DecValTok{1}\NormalTok{, }\AttributeTok{col =} \FunctionTok{heat.colors}\NormalTok{( }\DecValTok{50}\NormalTok{ ),}
       \AttributeTok{main =} \StringTok{"Posterior"}\NormalTok{, }\AttributeTok{xlab =} \StringTok{"distance, d"}\NormalTok{,}
       \AttributeTok{ylab =} \StringTok{"kappa, k"}\NormalTok{, }\AttributeTok{cex.main =} \FloatTok{2.0}\NormalTok{,}
       \AttributeTok{cex.lab =} \FloatTok{1.5}\NormalTok{, }\AttributeTok{cex.axis =} \FloatTok{1.5}\NormalTok{ )}
\FunctionTok{contour}\NormalTok{( }\AttributeTok{x =}\NormalTok{ d.v, }\AttributeTok{y =}\NormalTok{ k.v, }\AttributeTok{z =}\NormalTok{ Pos, }\AttributeTok{nlev =} \DecValTok{10}\NormalTok{,}
         \AttributeTok{drawlab =} \ConstantTok{FALSE}\NormalTok{, }\AttributeTok{add =} \ConstantTok{TRUE}\NormalTok{ )}
\end{Highlighting}
\end{Shaded}

\includegraphics{220303_Practical_session03_00_ex1_files/figure-latex/unnamed-chunk-8-1.pdf}

\hypertarget{part-2-markov-chain-monte-carlo-mcmc}{%
\section{PART 2: Markov Chain Monte Carlo
(MCMC)}\label{part-2-markov-chain-monte-carlo-mcmc}}

\hypertarget{introduction-1}{%
\subsection{Introduction}\label{introduction-1}}

Now, we want to obtain the posterior distribution by MCMC sampling. In
most practical problems, the marginal likelihood (aka constant \(z\))
cannot be calculated (either analytically or numerically), and so the
MCMC algorithm becomes necessary. In this case, we do not calculate the
posterior as:

\begin{quote}
\(f(\kappa,d|D)=\frac{f(d)f(\kappa)f(D|d,\kappa)}{z}\)
\end{quote}

Instead, we \textbf{approximate} the posterior to be the product of the
prior distribution/s and the likelihood. Following the example used
above, we have two priors (the prior on \(d\) and the prior on
\(\kappa\)) and the likelihood. To calculate this approximation, aka the
\textbf{unnormalised} posterior, we do the following:

\begin{quote}
\(f(\kappa,d|D)\propto f(d)f(\kappa)f(D|d,\kappa)\)
\end{quote}

Given that the priors for \(d\) and \(\kappa\) are Gamma distributions,
which depend on parameters \(\alpha\) and \(\beta\), i.e.,
\(\Gamma(\alpha,\beta)\)\ldots{}

\begin{quote}
E.g.: \(f(d)=\Gamma(d|2,20)\), if \(\alpha=2\) and \(\beta=20\).\\
\(f(\kappa)=\Gamma(d|2,.1)\), if \(\alpha=2\) and \(\beta=0.1\).
\end{quote}

\ldots{} we can write a function to compute the unnormalised posterior,
which we will later use when running the MCMC:

\begin{Shaded}
\begin{Highlighting}[]
\CommentTok{\# Define function that returns the logarithm of the unnormalised posterior:}
\CommentTok{\#                             f(d) * f(k) * f(D|d,k)}
\CommentTok{\# By, default we set the priors as:}
\CommentTok{\#                  f(d) = Gamma(d | 2, 20) and f(k) = Gamma(k | 2, .1)}
\CommentTok{\#}
\CommentTok{\# Arguments:}
\CommentTok{\#}
\CommentTok{\#   d     Numeric, value for the distance.}
\CommentTok{\#   k     Numeric, value for parameter kappa.}
\CommentTok{\#   n     Numeric, length of the alignment. Default: 948.}
\CommentTok{\#   ns    Numeric, total number of transitions. Default: 84.}
\CommentTok{\#   nv    Numeric, total number of transcersions. Default: 6.}
\CommentTok{\#   a.d.  Numeric, alpha value of the Gamma distribution that works as a prior}
\CommentTok{\#         for the distance (d). Default: 2.}
\CommentTok{\#   b.d.  Numeric, beta value pf the Gamma distribution that works as a prior}
\CommentTok{\#         for parameter distance (d). Default: 20.}
\CommentTok{\#   a.k.  Numeric, alpha value for the Gamma distribution that works as a prior}
\CommentTok{\#         for parameter kappa (k). Default: 2.}
\CommentTok{\#   b.k.  Numeric, beta value for the Gamma distribution that works as a prior}
\CommentTok{\#         for parameter kappa (k). Default: 0.1.}
\NormalTok{ulnPf }\OtherTok{\textless{}{-}} \ControlFlowTok{function}\NormalTok{( d, k, }\AttributeTok{n =} \DecValTok{948}\NormalTok{, }\AttributeTok{ns =} \DecValTok{84}\NormalTok{, }\AttributeTok{nv =} \DecValTok{6}\NormalTok{,}
                   \AttributeTok{a.d =} \DecValTok{2}\NormalTok{, }\AttributeTok{b.d =} \DecValTok{20}\NormalTok{, }\AttributeTok{a.k =} \DecValTok{2}\NormalTok{, }\AttributeTok{b.k =}\NormalTok{ .}\DecValTok{1}\NormalTok{ )\{}

  \CommentTok{\# The normalising constant in the prior densities can be ignored}
\NormalTok{  lnpriord }\OtherTok{\textless{}{-}}\NormalTok{ ( a.d }\SpecialCharTok{{-}} \DecValTok{1}\NormalTok{ )}\SpecialCharTok{*}\FunctionTok{log}\NormalTok{( d ) }\SpecialCharTok{{-}}\NormalTok{ b.d }\SpecialCharTok{*}\NormalTok{ d}
\NormalTok{  lnpriork }\OtherTok{\textless{}{-}}\NormalTok{ ( a.k }\SpecialCharTok{{-}} \DecValTok{1}\NormalTok{ )}\SpecialCharTok{*}\FunctionTok{log}\NormalTok{( k ) }\SpecialCharTok{{-}}\NormalTok{ b.k }\SpecialCharTok{*}\NormalTok{ k}

  \CommentTok{\# Define log{-}likelihood (K80 model)}
\NormalTok{  expd1 }\OtherTok{\textless{}{-}} \FunctionTok{exp}\NormalTok{( }\SpecialCharTok{{-}}\DecValTok{4}\SpecialCharTok{*}\NormalTok{d}\SpecialCharTok{/}\NormalTok{( k}\SpecialCharTok{+}\DecValTok{2}\NormalTok{ ) )}
\NormalTok{  expd2 }\OtherTok{\textless{}{-}} \FunctionTok{exp}\NormalTok{( }\SpecialCharTok{{-}}\DecValTok{2}\SpecialCharTok{*}\NormalTok{d}\SpecialCharTok{*}\NormalTok{( k}\SpecialCharTok{+}\DecValTok{1}\NormalTok{ )}\SpecialCharTok{/}\NormalTok{( k}\SpecialCharTok{+}\DecValTok{2}\NormalTok{ ) )}
\NormalTok{  p0 }\OtherTok{\textless{}{-}}\NormalTok{ .}\DecValTok{25} \SpecialCharTok{+}\NormalTok{ .}\DecValTok{25} \SpecialCharTok{*}\NormalTok{ expd1 }\SpecialCharTok{+}\NormalTok{ .}\DecValTok{5} \SpecialCharTok{*}\NormalTok{ expd2}
\NormalTok{  p1 }\OtherTok{\textless{}{-}}\NormalTok{ .}\DecValTok{25} \SpecialCharTok{+}\NormalTok{ .}\DecValTok{25} \SpecialCharTok{*}\NormalTok{ expd1 }\SpecialCharTok{{-}}\NormalTok{ .}\DecValTok{5} \SpecialCharTok{*}\NormalTok{ expd2}
\NormalTok{  p2 }\OtherTok{\textless{}{-}}\NormalTok{ .}\DecValTok{25} \SpecialCharTok{{-}}\NormalTok{ .}\DecValTok{25} \SpecialCharTok{*}\NormalTok{ expd1}
\NormalTok{  lnL }\OtherTok{\textless{}{-}}\NormalTok{ ( ( n }\SpecialCharTok{{-}}\NormalTok{ ns }\SpecialCharTok{{-}}\NormalTok{ nv ) }\SpecialCharTok{*} \FunctionTok{log}\NormalTok{( p0}\SpecialCharTok{/}\DecValTok{4}\NormalTok{ ) }\SpecialCharTok{+}\NormalTok{ ns }\SpecialCharTok{*} \FunctionTok{log}\NormalTok{( p1}\SpecialCharTok{/}\DecValTok{4}\NormalTok{ ) }\SpecialCharTok{+}\NormalTok{ nv }\SpecialCharTok{*} \FunctionTok{log}\NormalTok{( p2}\SpecialCharTok{/}\DecValTok{4}\NormalTok{ ) )}

  \CommentTok{\# Return unnormalised posterior (they are lnL, so }
  \CommentTok{\# you return their sum!)}
  \FunctionTok{return}\NormalTok{ ( lnpriord }\SpecialCharTok{+}\NormalTok{ lnpriork }\SpecialCharTok{+}\NormalTok{ lnL )}
\NormalTok{\}}
\end{Highlighting}
\end{Shaded}

\hypertarget{the-mcmc-algorithm}{%
\subsection{The MCMC algorithm}\label{the-mcmc-algorithm}}

Once we have established the function to compute the unnormalised
posterior, we need to define the function that we will use to run our
MCMC! The algorithm that we will implement in this function has five
main parts, which we detail below:

\begin{enumerate}
\def\labelenumi{\arabic{enumi}.}
\tightlist
\item
  Set initial states for \(d\) and \(\kappa\).\\
  Now, for \(n\) iterations:

  \begin{enumerate}
  \def\labelenumii{\arabic{enumii}.}
  \setcounter{enumii}{1}
  \tightlist
  \item
    Propose a new state \(d^{\*}\) (from an appropriate proposal
    density).\\
  \item
    Accept or reject the proposal with probability:
    \(\mathrm{min}(1,p(d^{\*})p(x|d^{\*})/p(d)p(x|d))\). If the proposal
    is accepted, then \(d=d^{\*}\). Otherwise, the same state for \(d\)
    is kept for the next iteration, \(d=d\).\\
  \item
    Save \(d\).\\
  \item
    Repeat steps 2-4 for \(\kappa\).\\
  \item
    Go to step 2.
  \end{enumerate}
\end{enumerate}

\begin{Shaded}
\begin{Highlighting}[]
\CommentTok{\# Define function with MCMC algorithm.}
\CommentTok{\#}
\CommentTok{\# Arguments:}
\CommentTok{\#}
\CommentTok{\#   init.d  Numeric, initial state value for parameter d.}
\CommentTok{\#   init.k  Numeric, initial state value for paramter k.}
\CommentTok{\#   N       Numeric, number of iterations that the MCMC will run.}
\CommentTok{\#   w.d     Numeric, width of the sliding{-}window proposal for d.}
\CommentTok{\#   w.k     Numeric, width of the sliding{-}window proposal for k.}
\NormalTok{mcmcf }\OtherTok{\textless{}{-}} \ControlFlowTok{function}\NormalTok{( init.d, init.k, N, w.d, w.k ) \{}

  \CommentTok{\# We keep the visited states (d, k) in sample.d and sample.k}
  \CommentTok{\# for easy plotting. In practical MCMC applications, these}
  \CommentTok{\# are usually written into a file. These two objects are numeric}
  \CommentTok{\# vectors of length N+1.}
\NormalTok{  sample.d }\OtherTok{\textless{}{-}}\NormalTok{ sample.k }\OtherTok{\textless{}{-}} \FunctionTok{numeric}\NormalTok{( N}\SpecialCharTok{+}\DecValTok{1}\NormalTok{ )}

  \CommentTok{\# STEP 1: Set initial parameter values to be used during the first}
  \CommentTok{\#         iteration of the MCMC.}
  \CommentTok{\# 1.1. Get initial values for parameters k and d. Save these values}
  \CommentTok{\#      in vectors sample.d and sample.k}
\NormalTok{  d }\OtherTok{\textless{}{-}}\NormalTok{ init.d;  sample.d[}\DecValTok{1}\NormalTok{] }\OtherTok{\textless{}{-}}\NormalTok{ init.d}
\NormalTok{  k }\OtherTok{\textless{}{-}}\NormalTok{ init.k;  sample.k[}\DecValTok{1}\NormalTok{] }\OtherTok{\textless{}{-}}\NormalTok{ init.k}
  \CommentTok{\# 1.2. Get unnormalised posterior}
\NormalTok{  ulnP  }\OtherTok{\textless{}{-}} \FunctionTok{ulnPf}\NormalTok{( }\AttributeTok{d =}\NormalTok{ d, }\AttributeTok{k =}\NormalTok{ k )}
  \CommentTok{\# 1.3. Initialise numeric vectors that will be used to keep track of}
  \CommentTok{\#      the number of times proposed values for each parameter,}
  \CommentTok{\#      d and k, have been accepted}
\NormalTok{  acc.d }\OtherTok{\textless{}{-}} \DecValTok{0}\NormalTok{; acc.k }\OtherTok{\textless{}{-}} \DecValTok{0}
  \CommentTok{\# 1.4. Start MCMC, which will run for N iterations}
  \ControlFlowTok{for}\NormalTok{ ( i }\ControlFlowTok{in} \DecValTok{1}\SpecialCharTok{:}\NormalTok{N )\{}

    \CommentTok{\# STEP 2: Propose a new state d*.}
    \CommentTok{\# We use a uniform sliding window of width w with reflection}
    \CommentTok{\# to propose new values d* and k*}
    \CommentTok{\# 2.1. Propose d* and accept/reject the proposal}
\NormalTok{    dprop }\OtherTok{\textless{}{-}}\NormalTok{ d }\SpecialCharTok{+} \FunctionTok{runif}\NormalTok{( }\AttributeTok{n =} \DecValTok{1}\NormalTok{, }\AttributeTok{min =} \SpecialCharTok{{-}}\NormalTok{w.d}\SpecialCharTok{/}\DecValTok{2}\NormalTok{, }\AttributeTok{max =}\NormalTok{ w.d}\SpecialCharTok{/}\DecValTok{2}\NormalTok{ )}
    \CommentTok{\# 2.2. Reflect if dprop is negative}
    \ControlFlowTok{if}\NormalTok{ ( dprop }\SpecialCharTok{\textless{}} \DecValTok{0}\NormalTok{ ) dprop }\OtherTok{\textless{}{-}} \SpecialCharTok{{-}}\NormalTok{dprop}
    \CommentTok{\# 2.3. Compute unnormalised posterior}
\NormalTok{    ulnPprop }\OtherTok{\textless{}{-}} \FunctionTok{ulnPf}\NormalTok{( }\AttributeTok{d =}\NormalTok{ dprop, }\AttributeTok{k =}\NormalTok{ k )}
\NormalTok{    lnalpha  }\OtherTok{\textless{}{-}}\NormalTok{ ulnPprop }\SpecialCharTok{{-}}\NormalTok{ ulnP}

    \CommentTok{\# STEP 3: Accept or reject the proposal:}
    \CommentTok{\#            if ru \textless{} alpha accept proposed d*}
    \CommentTok{\#            else reject and stay where we are}
    \ControlFlowTok{if}\NormalTok{ ( lnalpha }\SpecialCharTok{\textgreater{}} \DecValTok{0} \SpecialCharTok{||} \FunctionTok{runif}\NormalTok{( }\AttributeTok{n =} \DecValTok{1}\NormalTok{ ) }\SpecialCharTok{\textless{}} \FunctionTok{exp}\NormalTok{( lnalpha ) )\{}
\NormalTok{      d      }\OtherTok{\textless{}{-}}\NormalTok{ dprop}
\NormalTok{      ulnP   }\OtherTok{\textless{}{-}}\NormalTok{ ulnPprop}
\NormalTok{      acc.d  }\OtherTok{\textless{}{-}}\NormalTok{ acc.d }\SpecialCharTok{+} \DecValTok{1}
\NormalTok{    \}}

    \CommentTok{\# STEP 4: Repeat steps 2{-}3 to propose a new state k*.}
    \CommentTok{\# 4.1. Propose k* and accept/reject the proposal}
\NormalTok{    kprop }\OtherTok{\textless{}{-}}\NormalTok{ k }\SpecialCharTok{+} \FunctionTok{runif}\NormalTok{( }\AttributeTok{n =} \DecValTok{1}\NormalTok{, }\AttributeTok{min =} \SpecialCharTok{{-}}\NormalTok{w.k}\SpecialCharTok{/}\DecValTok{2}\NormalTok{, }\AttributeTok{max =}\NormalTok{ w.k}\SpecialCharTok{/}\DecValTok{2}\NormalTok{ )}
    \CommentTok{\# 4.2. Reflect if kprop is negative}
    \ControlFlowTok{if}\NormalTok{ ( kprop }\SpecialCharTok{\textless{}} \DecValTok{0}\NormalTok{ ) kprop }\OtherTok{\textless{}{-}} \SpecialCharTok{{-}}\NormalTok{kprop}
    \CommentTok{\# 4.3. Compute unnormalised posterior}
\NormalTok{    ulnPprop }\OtherTok{\textless{}{-}} \FunctionTok{ulnPf}\NormalTok{( }\AttributeTok{d =}\NormalTok{ d, }\AttributeTok{k =}\NormalTok{ kprop )}
\NormalTok{    lnalpha  }\OtherTok{\textless{}{-}}\NormalTok{ ulnPprop }\SpecialCharTok{{-}}\NormalTok{ ulnP}
    \CommentTok{\# 4.4. Accept/reject proposal:}
    \CommentTok{\#          if ru \textless{} alpha accept proposed k*}
    \CommentTok{\#          else reject and stay where we are}
    \ControlFlowTok{if}\NormalTok{ ( lnalpha }\SpecialCharTok{\textgreater{}} \DecValTok{0} \SpecialCharTok{||} \FunctionTok{runif}\NormalTok{( }\AttributeTok{n =} \DecValTok{1}\NormalTok{ ) }\SpecialCharTok{\textless{}} \FunctionTok{exp}\NormalTok{( lnalpha ) )\{}
\NormalTok{      k     }\OtherTok{\textless{}{-}}\NormalTok{ kprop}
\NormalTok{      ulnP  }\OtherTok{\textless{}{-}}\NormalTok{ ulnPprop}
\NormalTok{      acc.k }\OtherTok{\textless{}{-}}\NormalTok{ acc.k }\SpecialCharTok{+} \DecValTok{1}
\NormalTok{    \}}

    \CommentTok{\# STEP 5: Save chain state for each parameter so we can later}
    \CommentTok{\#         plot the corresponding histograms}
\NormalTok{    sample.d[i}\SpecialCharTok{+}\DecValTok{1}\NormalTok{] }\OtherTok{\textless{}{-}}\NormalTok{ d}
\NormalTok{    sample.k[i}\SpecialCharTok{+}\DecValTok{1}\NormalTok{] }\OtherTok{\textless{}{-}}\NormalTok{ k}
\NormalTok{  \}}

  \CommentTok{\# Print out the proportion of times}
  \CommentTok{\# the proposals were accepted}
  \FunctionTok{cat}\NormalTok{( }\StringTok{"Acceptance proportions:}\SpecialCharTok{\textbackslash{}n}\StringTok{"}\NormalTok{, }\StringTok{"d: "}\NormalTok{, acc.d}\SpecialCharTok{/}\NormalTok{N, }\StringTok{" | k: "}\NormalTok{, acc.k}\SpecialCharTok{/}\NormalTok{N, }\StringTok{"}\SpecialCharTok{\textbackslash{}n}\StringTok{"}\NormalTok{ )}

  \CommentTok{\# Return vector of d and k visited during MCMC}
  \FunctionTok{return}\NormalTok{( }\FunctionTok{list}\NormalTok{( }\AttributeTok{d =}\NormalTok{ sample.d, }\AttributeTok{k =}\NormalTok{ sample.k ) )}

\NormalTok{\}}
\end{Highlighting}
\end{Shaded}

Before proceeding with the next steps, we can set a seed number so we
can later reproduce the results that we get when running the MCMCs that
you will see below. You can ommit running the next command if you want
to get different results each time you run this tutorial:

\begin{Shaded}
\begin{Highlighting}[]
\FunctionTok{set.seed}\NormalTok{( }\DecValTok{12345}\NormalTok{ )}
\end{Highlighting}
\end{Shaded}

\hypertarget{running-an-mcmc}{%
\subsection{Running an MCMC}\label{running-an-mcmc}}

Before we run our MCMC function, we might want to estimate how long this
might take. The function \texttt{system.time()} can be used for this
purpose:

\begin{Shaded}
\begin{Highlighting}[]
\CommentTok{\# Test run{-}time}
\FunctionTok{system.time}\NormalTok{( }\FunctionTok{mcmcf}\NormalTok{( }\AttributeTok{init.d =} \FloatTok{0.2}\NormalTok{, }\AttributeTok{init.k =} \DecValTok{20}\NormalTok{, }\AttributeTok{N =} \FloatTok{1e4}\NormalTok{,}
                    \AttributeTok{w.d =} \FloatTok{0.12}\NormalTok{, }\AttributeTok{w.k =} \DecValTok{180}\NormalTok{ ) )}
\end{Highlighting}
\end{Shaded}

\begin{verbatim}
## Acceptance proportions:
##  d:  0.3001  | k:  0.3109
\end{verbatim}

\begin{verbatim}
##    user  system elapsed 
##   0.270   0.029   0.298
\end{verbatim}

Once we have an estimate of the time it can take us to run an MCMC with
specific settings (see R code above), we can run our function and save
the output in an object, which we will later have to inspect:

\begin{Shaded}
\begin{Highlighting}[]
\CommentTok{\# Run MCMC and save output}
\NormalTok{dk.mcmc }\OtherTok{\textless{}{-}} \FunctionTok{mcmcf}\NormalTok{( }\AttributeTok{init.d =} \FloatTok{0.2}\NormalTok{, }\AttributeTok{init.k =} \DecValTok{20}\NormalTok{, }\AttributeTok{N =} \FloatTok{1e4}\NormalTok{,}
                  \AttributeTok{w.d =} \FloatTok{0.12}\NormalTok{, }\AttributeTok{w.k =} \DecValTok{180}\NormalTok{ )}
\end{Highlighting}
\end{Shaded}

\begin{verbatim}
## Acceptance proportions:
##  d:  0.3013  | k:  0.3069
\end{verbatim}

\hypertarget{mcmc-diagnostics-using-trace-plots}{%
\subsection{MCMC diagnostics: using trace
plots}\label{mcmc-diagnostics-using-trace-plots}}

The next step is very important: you need to make sure that your chain
has converged and that there have been no issues during the sampling.
For this purpose, we can use visual diagnostics such as plotting the
traces of the sampled values during the \(n\) iterations the chain ran:

\begin{Shaded}
\begin{Highlighting}[]
\CommentTok{\# Plot traces of the values sampled for each parameter}
\FunctionTok{par}\NormalTok{( }\AttributeTok{mfrow =} \FunctionTok{c}\NormalTok{( }\DecValTok{1}\NormalTok{,}\DecValTok{3}\NormalTok{ ) )}
\CommentTok{\# Plot trace for parameter d}
\FunctionTok{plot}\NormalTok{( }\AttributeTok{x =}\NormalTok{ dk.mcmc}\SpecialCharTok{$}\NormalTok{d, }\AttributeTok{ty =} \StringTok{\textquotesingle{}l\textquotesingle{}}\NormalTok{, }\AttributeTok{xlab =} \StringTok{"Iteration"}\NormalTok{,}
      \AttributeTok{ylab =} \StringTok{"d"}\NormalTok{, }\AttributeTok{main =} \StringTok{"Trace of d"}\NormalTok{ )}
\CommentTok{\# Plot trace for parameter k}
\FunctionTok{plot}\NormalTok{( }\AttributeTok{x =}\NormalTok{ dk.mcmc}\SpecialCharTok{$}\NormalTok{k, }\AttributeTok{ty =} \StringTok{\textquotesingle{}l\textquotesingle{}}\NormalTok{, }\AttributeTok{xlab =} \StringTok{"Iteration"}\NormalTok{,}
      \AttributeTok{ylab =} \StringTok{"k"}\NormalTok{, }\AttributeTok{main =} \StringTok{"Trace of k"}\NormalTok{ )}
\CommentTok{\# Plot the joint sample of d and k (points sampled from posterior surface)}
\FunctionTok{plot}\NormalTok{( }\AttributeTok{x =}\NormalTok{ dk.mcmc}\SpecialCharTok{$}\NormalTok{d, }\AttributeTok{y =}\NormalTok{ dk.mcmc}\SpecialCharTok{$}\NormalTok{k, }\AttributeTok{pch =} \StringTok{\textquotesingle{}.\textquotesingle{}}\NormalTok{, }\AttributeTok{xlab =} \StringTok{"d"}\NormalTok{,}
      \AttributeTok{ylab =} \StringTok{"k"}\NormalTok{, }\AttributeTok{main =} \StringTok{"Joint of d and k"}\NormalTok{ )}
\end{Highlighting}
\end{Shaded}

\includegraphics{220303_Practical_session03_00_ex1_files/figure-latex/unnamed-chunk-14-1.pdf}

\emph{\textbf{QUESTION 1}} What would you say about these chains? Have
they converged? As an additional exercise, you can change the starting
parameter values, run an MCMC with the new settings, and then check for
chain convergence. Does something change with your new settings?

\hypertarget{part-3-efficiency-of-the-mcmc-chain}{%
\section{PART 3: Efficiency of the MCMC
chain}\label{part-3-efficiency-of-the-mcmc-chain}}

\hypertarget{introduction-2}{%
\subsection{Introduction}\label{introduction-2}}

Remember that values sampled in an MCMC are autocorrelated because the
proposed state for the next iteration is either the current state (i.e.,
the new proposed state during the current iteration is rejected, and
thus the current state is kept for the next iteration) or a modification
of it (i.e., the new proposed state, based on a distribution that uses
the current state, is accepted to be used in the next iteration).

In addition, the efficiency of an MCMC chain is closely related to the
autocorrelation. Intuitively, if the autocorrelation is high, the chain
will be inefficient, i.e., we will need to run the chain for a long time
to obtain a good approximation to the posterior distribution.

The efficiency of a chain is defined as: \textgreater{}
eff=1/(1+2(r1+r2+r3+\ldots)) where \(r_{i}\) is the correlation for lag
\(i\).

We are going to go through different examples with which we can evaluate
chain efficiency.

\hypertarget{autocorrelation-function-acf-plots}{%
\subsection{Autocorrelation Function (ACF)
plots}\label{autocorrelation-function-acf-plots}}

First, we will run a very long chain and plot the corresponding
autocorrelation function (ACF) for each parameter:

\begin{Shaded}
\begin{Highlighting}[]
\CommentTok{\# Run n=1e6 iterations}
\NormalTok{dk.mcmc2 }\OtherTok{\textless{}{-}} \FunctionTok{mcmcf}\NormalTok{( }\AttributeTok{init.d =} \FloatTok{0.2}\NormalTok{, }\AttributeTok{init.k =} \DecValTok{20}\NormalTok{, }\AttributeTok{N =} \FloatTok{1e6}\NormalTok{,}
                   \AttributeTok{w.d =}\NormalTok{ .}\DecValTok{12}\NormalTok{, }\AttributeTok{w.k =} \DecValTok{180}\NormalTok{ )}
\end{Highlighting}
\end{Shaded}

\begin{verbatim}
## Acceptance proportions:
##  d:  0.300366  | k:  0.305855
\end{verbatim}

\begin{Shaded}
\begin{Highlighting}[]
\CommentTok{\# Plot ACF for each parameter}
\FunctionTok{par}\NormalTok{( }\AttributeTok{mfrow =} \FunctionTok{c}\NormalTok{( }\DecValTok{1}\NormalTok{,}\DecValTok{2}\NormalTok{ ) )}
\FunctionTok{acf}\NormalTok{( }\AttributeTok{x =}\NormalTok{ dk.mcmc2}\SpecialCharTok{$}\NormalTok{d )}
\FunctionTok{acf}\NormalTok{( }\AttributeTok{x =}\NormalTok{ dk.mcmc2}\SpecialCharTok{$}\NormalTok{k )}
\end{Highlighting}
\end{Shaded}

\includegraphics{220303_Practical_session03_00_ex1_files/figure-latex/unnamed-chunk-15-1.pdf}

\emph{\textbf{QUESTION 1}} What can you tell about this MCMC run by
looking at these ACF plots?

\hypertarget{calculating-chain-efficiency-values}{%
\subsection{Calculating chain efficiency
values}\label{calculating-chain-efficiency-values}}

Apart from calculating and plotting the ACF, we can also calculate chain
efficiency. For this purpose, we use the following function:

\begin{Shaded}
\begin{Highlighting}[]
\CommentTok{\# Define efficiency function}
\CommentTok{\#}
\CommentTok{\# Arguments:}
\CommentTok{\#  acf  Numeric, autocorrelation value}
\NormalTok{eff }\OtherTok{\textless{}{-}} \ControlFlowTok{function}\NormalTok{( acf ) }\DecValTok{1} \SpecialCharTok{/}\NormalTok{ ( }\DecValTok{1} \SpecialCharTok{+} \DecValTok{2} \SpecialCharTok{*} \FunctionTok{sum}\NormalTok{( acf}\SpecialCharTok{$}\NormalTok{acf[}\SpecialCharTok{{-}}\DecValTok{1}\NormalTok{] ) )}
\end{Highlighting}
\end{Shaded}

As we have saved the sampled values from the chain previously run in
object \texttt{dk.mcmc2}, we can now compute this chain's efficiency:

\begin{Shaded}
\begin{Highlighting}[]
\CommentTok{\# Compute efficiency}
\FunctionTok{eff}\NormalTok{( }\AttributeTok{acf =} \FunctionTok{acf}\NormalTok{( dk.mcmc2}\SpecialCharTok{$}\NormalTok{d ) )}
\end{Highlighting}
\end{Shaded}

\includegraphics{220303_Practical_session03_00_ex1_files/figure-latex/unnamed-chunk-17-1.pdf}

\begin{verbatim}
## [1] 0.2243995
\end{verbatim}

\begin{Shaded}
\begin{Highlighting}[]
\FunctionTok{eff}\NormalTok{( }\AttributeTok{acf =} \FunctionTok{acf}\NormalTok{( dk.mcmc2}\SpecialCharTok{$}\NormalTok{k ) )}
\end{Highlighting}
\end{Shaded}

\includegraphics{220303_Practical_session03_00_ex1_files/figure-latex/unnamed-chunk-17-2.pdf}

\begin{verbatim}
## [1] 0.204568
\end{verbatim}

\emph{\textbf{QUESTION 1}} What can you tell about this chain with
regards to the efficiency values you have just computed?

\hypertarget{comparing-efficient-vs-inefficient-mcmcs}{%
\subsection{Comparing efficient VS inefficient
MCMCs}\label{comparing-efficient-vs-inefficient-mcmcs}}

\hypertarget{example-1}{%
\subsubsection{Example 1}\label{example-1}}

We will now run another MCMC but, this time, using a proposal density
with a too large step size for parameter d and another with a too small
step size for parameter \(\kappa\):

\begin{Shaded}
\begin{Highlighting}[]
\NormalTok{dk.mcmc3 }\OtherTok{\textless{}{-}} \FunctionTok{mcmcf}\NormalTok{( }\AttributeTok{init.d =} \FloatTok{0.2}\NormalTok{, }\AttributeTok{init.k =} \DecValTok{20}\NormalTok{, }\AttributeTok{N =} \FloatTok{1e4}\NormalTok{,}
                   \AttributeTok{w.d =} \DecValTok{3}\NormalTok{, }\AttributeTok{w.k =} \DecValTok{5}\NormalTok{ )}
\end{Highlighting}
\end{Shaded}

\begin{verbatim}
## Acceptance proportions:
##  d:  0.0208  | k:  0.9362
\end{verbatim}

When the MCMC has finished, we can plot the corresponding traces for
each parameter:

\begin{Shaded}
\begin{Highlighting}[]
\CommentTok{\# Plot traces for each parameter.}
\FunctionTok{par}\NormalTok{( }\AttributeTok{mfrow =} \FunctionTok{c}\NormalTok{( }\DecValTok{1}\NormalTok{,}\DecValTok{2}\NormalTok{ ) )}
\FunctionTok{plot}\NormalTok{( }\AttributeTok{x =}\NormalTok{ dk.mcmc3}\SpecialCharTok{$}\NormalTok{d, }\AttributeTok{ty =} \StringTok{\textquotesingle{}l\textquotesingle{}}\NormalTok{, }\AttributeTok{main =} \StringTok{"Trace of d"}\NormalTok{, }\AttributeTok{cex.main =} \FloatTok{2.0}\NormalTok{,}
      \AttributeTok{cex.lab =} \FloatTok{1.5}\NormalTok{, }\AttributeTok{cex.axis =} \FloatTok{1.5}\NormalTok{, }\AttributeTok{ylab =} \StringTok{"d"}\NormalTok{ )}
\FunctionTok{plot}\NormalTok{( }\AttributeTok{x =}\NormalTok{ dk.mcmc3}\SpecialCharTok{$}\NormalTok{k, }\AttributeTok{ty =} \StringTok{\textquotesingle{}l\textquotesingle{}}\NormalTok{, }\AttributeTok{main =} \StringTok{"Trace of k"}\NormalTok{, }\AttributeTok{cex.main =} \FloatTok{2.0}\NormalTok{,}
      \AttributeTok{cex.lab =} \FloatTok{1.5}\NormalTok{, }\AttributeTok{cex.axis =} \FloatTok{1.5}\NormalTok{, }\AttributeTok{ylab =} \StringTok{"k"}\NormalTok{ )}
\end{Highlighting}
\end{Shaded}

\includegraphics{220303_Practical_session03_00_ex1_files/figure-latex/unnamed-chunk-19-1.pdf}

\emph{\textbf{QUESTION 1}} What can you tell about the chain efficiency
when using these parameters?

\hypertarget{example-2}{%
\subsubsection{Example 2}\label{example-2}}

Now, we run the chain longer but keep the same starting values for the
rest of the parameters. Then, we compute the chain efficiency for each
parameter:

\begin{Shaded}
\begin{Highlighting}[]
\CommentTok{\# Run MCMC}
\NormalTok{dk.mcmc4 }\OtherTok{\textless{}{-}} \FunctionTok{mcmcf}\NormalTok{( }\AttributeTok{init.d =} \FloatTok{0.2}\NormalTok{, }\AttributeTok{init.k =} \DecValTok{20}\NormalTok{, }\AttributeTok{N =} \FloatTok{1e6}\NormalTok{,}
                   \AttributeTok{w.d =} \DecValTok{3}\NormalTok{, }\AttributeTok{w.k =} \DecValTok{5}\NormalTok{ )}
\end{Highlighting}
\end{Shaded}

\begin{verbatim}
## Acceptance proportions:
##  d:  0.024503  | k:  0.941828
\end{verbatim}

\begin{Shaded}
\begin{Highlighting}[]
\CommentTok{\# Compute efficiency values}
\FunctionTok{eff}\NormalTok{( }\AttributeTok{acf =} \FunctionTok{acf}\NormalTok{( dk.mcmc4}\SpecialCharTok{$}\NormalTok{d, }\AttributeTok{lag.max =} \FloatTok{2e3}\NormalTok{ ) )}
\end{Highlighting}
\end{Shaded}

\includegraphics{220303_Practical_session03_00_ex1_files/figure-latex/unnamed-chunk-20-1.pdf}

\begin{verbatim}
## [1] 0.01546219
\end{verbatim}

\begin{Shaded}
\begin{Highlighting}[]
\FunctionTok{eff}\NormalTok{( }\AttributeTok{acf =} \FunctionTok{acf}\NormalTok{( dk.mcmc4}\SpecialCharTok{$}\NormalTok{k, }\AttributeTok{lag.max =} \FloatTok{2e3}\NormalTok{ ) )}
\end{Highlighting}
\end{Shaded}

\includegraphics{220303_Practical_session03_00_ex1_files/figure-latex/unnamed-chunk-20-2.pdf}

\begin{verbatim}
## [1] 0.003721209
\end{verbatim}

\begin{Shaded}
\begin{Highlighting}[]
\CommentTok{\# Plot the traces for efficient (part 2) and inefficient chains.}
\FunctionTok{par}\NormalTok{( }\AttributeTok{mfrow =} \FunctionTok{c}\NormalTok{( }\DecValTok{2}\NormalTok{,}\DecValTok{2}\NormalTok{ ) )}
\FunctionTok{plot}\NormalTok{( dk.mcmc}\SpecialCharTok{$}\NormalTok{d, }\AttributeTok{ty =} \StringTok{\textquotesingle{}l\textquotesingle{}}\NormalTok{, }\AttributeTok{las =} \DecValTok{1}\NormalTok{, }\AttributeTok{ylim =} \FunctionTok{c}\NormalTok{( .}\DecValTok{05}\NormalTok{,.}\DecValTok{2}\NormalTok{ ),}
      \AttributeTok{main =} \StringTok{"Trace of d, efficient chain"}\NormalTok{, }\AttributeTok{xlab =} \StringTok{\textquotesingle{}\textquotesingle{}}\NormalTok{,}
      \AttributeTok{ylab =} \StringTok{"Distance, d"}\NormalTok{, }\AttributeTok{cex.main =} \FloatTok{2.0}\NormalTok{, }\AttributeTok{cex.lab =} \FloatTok{1.5}\NormalTok{ )}
\FunctionTok{plot}\NormalTok{( dk.mcmc3}\SpecialCharTok{$}\NormalTok{d, }\AttributeTok{ty =} \StringTok{\textquotesingle{}l\textquotesingle{}}\NormalTok{, }\AttributeTok{las =} \DecValTok{1}\NormalTok{, }\AttributeTok{ylim =} \FunctionTok{c}\NormalTok{( .}\DecValTok{05}\NormalTok{,.}\DecValTok{2}\NormalTok{ ),}
      \AttributeTok{main =} \StringTok{"Trace of d, inefficient chain"}\NormalTok{, }\AttributeTok{xlab=}\StringTok{\textquotesingle{}\textquotesingle{}}\NormalTok{,}
      \AttributeTok{ylab =} \StringTok{\textquotesingle{}\textquotesingle{}}\NormalTok{, }\AttributeTok{cex.main =} \FloatTok{2.0}\NormalTok{, }\AttributeTok{cex.lab =} \FloatTok{1.5}\NormalTok{ )}
\FunctionTok{plot}\NormalTok{( dk.mcmc}\SpecialCharTok{$}\NormalTok{k, }\AttributeTok{ty =} \StringTok{\textquotesingle{}l\textquotesingle{}}\NormalTok{, }\AttributeTok{las =} \DecValTok{1}\NormalTok{, }\AttributeTok{ylim =} \FunctionTok{c}\NormalTok{( }\DecValTok{0}\NormalTok{,}\DecValTok{100}\NormalTok{ ),}
      \AttributeTok{main =} \StringTok{"Trace of k, efficient chain"}\NormalTok{,}
      \AttributeTok{xlab =} \StringTok{\textquotesingle{}\textquotesingle{}}\NormalTok{, }\AttributeTok{ylab =} \StringTok{"ts/tv ratio, k"}\NormalTok{,}
      \AttributeTok{cex.main =} \FloatTok{2.0}\NormalTok{, }\AttributeTok{cex.lab =} \FloatTok{1.5}\NormalTok{ )}
\FunctionTok{plot}\NormalTok{( dk.mcmc3}\SpecialCharTok{$}\NormalTok{k, }\AttributeTok{ty =} \StringTok{\textquotesingle{}l\textquotesingle{}}\NormalTok{, }\AttributeTok{las =} \DecValTok{1}\NormalTok{, }\AttributeTok{ylim =} \FunctionTok{c}\NormalTok{( }\DecValTok{0}\NormalTok{,}\DecValTok{100}\NormalTok{ ),}
      \AttributeTok{main =} \StringTok{"Trace of k, inefficient chain"}\NormalTok{,}
      \AttributeTok{xlab =} \StringTok{\textquotesingle{}\textquotesingle{}}\NormalTok{, }\AttributeTok{ylab =} \StringTok{\textquotesingle{}\textquotesingle{}}\NormalTok{, }\AttributeTok{cex.main =} \FloatTok{2.0}\NormalTok{, }\AttributeTok{cex.lab =} \FloatTok{1.5}\NormalTok{ )}
\end{Highlighting}
\end{Shaded}

\includegraphics{220303_Practical_session03_00_ex1_files/figure-latex/unnamed-chunk-20-3.pdf}

\emph{\textbf{QUESTION 1}} What differences can you observe after
running the chain longer?

\hypertarget{part-4-checking-for-convergence}{%
\section{PART 4: Checking for
convergence}\label{part-4-checking-for-convergence}}

Last, we will learn more about the concept of burn-in and its effect on
chain convergence.

\hypertarget{example-1-1}{%
\subsection{Example 1}\label{example-1-1}}

We will run two different chains: one with a high starting value for
parameters d and \(\kappa\) and another with a low starting value for
these two parameters:

\begin{Shaded}
\begin{Highlighting}[]
\CommentTok{\# Run MCMCs with high/low starting values for parameters d and k.}
\NormalTok{dk.mcmc.l }\OtherTok{\textless{}{-}} \FunctionTok{mcmcf}\NormalTok{( }\AttributeTok{init.d =} \FloatTok{0.01}\NormalTok{, }\AttributeTok{init.k =} \DecValTok{20}\NormalTok{, }\AttributeTok{N =} \FloatTok{1e4}\NormalTok{,}
                    \AttributeTok{w.d =}\NormalTok{ .}\DecValTok{12}\NormalTok{, }\AttributeTok{w.k =} \DecValTok{180}\NormalTok{ )}
\end{Highlighting}
\end{Shaded}

\begin{verbatim}
## Acceptance proportions:
##  d:  0.2964  | k:  0.2972
\end{verbatim}

\begin{Shaded}
\begin{Highlighting}[]
\NormalTok{dk.mcmc.h }\OtherTok{\textless{}{-}} \FunctionTok{mcmcf}\NormalTok{( }\AttributeTok{init.d =} \FloatTok{0.4}\NormalTok{, }\AttributeTok{init.k =} \DecValTok{20}\NormalTok{, }\AttributeTok{N =} \FloatTok{1e4}\NormalTok{,}
                    \AttributeTok{w.d =}\NormalTok{ .}\DecValTok{12}\NormalTok{, }\AttributeTok{w.k =} \DecValTok{180}\NormalTok{ )}
\end{Highlighting}
\end{Shaded}

\begin{verbatim}
## Acceptance proportions:
##  d:  0.3057  | k:  0.3083
\end{verbatim}

Now, we can compute the mean and the standard deviation of parameter d.
Below, we show you how this can be done when using the ``low'' chain,
although we could have used the ``high'' chain too:

\begin{Shaded}
\begin{Highlighting}[]
\CommentTok{\# Compute mean and sd for d}
\NormalTok{mean.d }\OtherTok{\textless{}{-}} \FunctionTok{mean}\NormalTok{( dk.mcmc.l}\SpecialCharTok{$}\NormalTok{d )}
\NormalTok{sd.d   }\OtherTok{\textless{}{-}} \FunctionTok{sd}\NormalTok{( dk.mcmc.l}\SpecialCharTok{$}\NormalTok{d )}
\end{Highlighting}
\end{Shaded}

Now, we can plot the two chains, ``low'' and ``high'', to observe how
the chains move from either the high or low starting values towards
the\\
stationary phase (the area within the dashed lines). The area before it
reaches stationarity is what we call the ``burn-in'' phase:

\begin{Shaded}
\begin{Highlighting}[]
\CommentTok{\# Plot the two chains }
\FunctionTok{plot}\NormalTok{( dk.mcmc.l}\SpecialCharTok{$}\NormalTok{d, }\AttributeTok{xlim =} \FunctionTok{c}\NormalTok{( }\DecValTok{1}\NormalTok{,}\DecValTok{200}\NormalTok{ ), }\AttributeTok{ylim =} \FunctionTok{c}\NormalTok{( }\DecValTok{0}\NormalTok{,}\FloatTok{0.4}\NormalTok{ ), }\AttributeTok{ty =} \StringTok{"l"}\NormalTok{ )}
\FunctionTok{lines}\NormalTok{( dk.mcmc.h}\SpecialCharTok{$}\NormalTok{d, }\AttributeTok{col =} \StringTok{"red"}\NormalTok{ )}
\CommentTok{\# Plot a horizontal dashed line to indicate (approximately)}
\CommentTok{\# the 95\% CI.}
\FunctionTok{abline}\NormalTok{( }\AttributeTok{h =}\NormalTok{ mean.d }\SpecialCharTok{+} \DecValTok{2} \SpecialCharTok{*} \FunctionTok{c}\NormalTok{( }\SpecialCharTok{{-}}\NormalTok{sd.d, sd.d ), }\AttributeTok{lty =} \DecValTok{2}\NormalTok{ )}
\end{Highlighting}
\end{Shaded}

\includegraphics{220303_Practical_session03_00_ex1_files/figure-latex/unnamed-chunk-23-1.pdf}

\hypertarget{example-2-1}{%
\subsection{Example 2}\label{example-2-1}}

We are going to run two chains with different starting values so we can
compare their efficiency:

\begin{Shaded}
\begin{Highlighting}[]
\CommentTok{\# Run an efficient chain (i.e., good proposal step sizes)}
\NormalTok{dk.mcmc.b }\OtherTok{\textless{}{-}} \FunctionTok{mcmcf}\NormalTok{( }\AttributeTok{init.d =} \FloatTok{0.05}\NormalTok{, }\AttributeTok{init.k =} \DecValTok{5}\NormalTok{, }\AttributeTok{N =} \FloatTok{1e4}\NormalTok{,}
                    \AttributeTok{w.d =}\NormalTok{ .}\DecValTok{12}\NormalTok{, }\AttributeTok{w.k =} \DecValTok{180}\NormalTok{ )}
\end{Highlighting}
\end{Shaded}

\begin{verbatim}
## Acceptance proportions:
##  d:  0.299  | k:  0.3069
\end{verbatim}

\begin{Shaded}
\begin{Highlighting}[]
\CommentTok{\# Run an inefficient chain (i.e., bad proposal step sizes)}
\NormalTok{dk.mcmc3.b }\OtherTok{\textless{}{-}} \FunctionTok{mcmcf}\NormalTok{( }\AttributeTok{init.d  =} \FloatTok{0.05}\NormalTok{, }\AttributeTok{init.k =} \DecValTok{5}\NormalTok{, }\AttributeTok{N =} \FloatTok{1e4}\NormalTok{,}
                     \AttributeTok{w.d =} \DecValTok{3}\NormalTok{, }\AttributeTok{w.k =} \DecValTok{5}\NormalTok{ )}
\end{Highlighting}
\end{Shaded}

\begin{verbatim}
## Acceptance proportions:
##  d:  0.0226  | k:  0.9377
\end{verbatim}

\begin{Shaded}
\begin{Highlighting}[]
\CommentTok{\# Plot and compare histograms}
\CommentTok{\# Set breaking points for the plot}
\NormalTok{bks }\OtherTok{\textless{}{-}} \FunctionTok{seq}\NormalTok{(}\AttributeTok{from=}\DecValTok{0}\NormalTok{, }\AttributeTok{to=}\DecValTok{150}\NormalTok{, }\AttributeTok{by=}\DecValTok{1}\NormalTok{)}
\CommentTok{\# Start plotting}
\FunctionTok{par}\NormalTok{( }\AttributeTok{mfrow =} \FunctionTok{c}\NormalTok{( }\DecValTok{1}\NormalTok{,}\DecValTok{2}\NormalTok{ ) )}
\FunctionTok{hist}\NormalTok{( }\AttributeTok{x =}\NormalTok{ dk.mcmc.b}\SpecialCharTok{$}\NormalTok{k, }\AttributeTok{prob =} \ConstantTok{TRUE}\NormalTok{, }\AttributeTok{breaks =}\NormalTok{ bks, }\AttributeTok{border =} \ConstantTok{NA}\NormalTok{,}
      \AttributeTok{col =} \FunctionTok{rgb}\NormalTok{( }\DecValTok{0}\NormalTok{, }\DecValTok{0}\NormalTok{, }\DecValTok{1}\NormalTok{, .}\DecValTok{5}\NormalTok{ ), }\AttributeTok{las =} \DecValTok{1}\NormalTok{, }\AttributeTok{xlab =} \StringTok{"kappa"}\NormalTok{,}
      \AttributeTok{xlim =} \FunctionTok{c}\NormalTok{( }\DecValTok{0}\NormalTok{,}\DecValTok{100}\NormalTok{ ), }\AttributeTok{ylim =} \FunctionTok{c}\NormalTok{( }\DecValTok{0}\NormalTok{,.}\DecValTok{055}\NormalTok{ ) )}
\FunctionTok{hist}\NormalTok{( }\AttributeTok{x =}\NormalTok{ dk.mcmc}\SpecialCharTok{$}\NormalTok{k, }\AttributeTok{prob=}\ConstantTok{TRUE}\NormalTok{, }\AttributeTok{breaks=}\NormalTok{bks, }\AttributeTok{border=}\ConstantTok{NA}\NormalTok{,}
      \AttributeTok{col=}\FunctionTok{rgb}\NormalTok{(.}\DecValTok{5}\NormalTok{, .}\DecValTok{5}\NormalTok{, .}\DecValTok{5}\NormalTok{, .}\DecValTok{5}\NormalTok{), }\AttributeTok{add=}\ConstantTok{TRUE}\NormalTok{)}
\FunctionTok{hist}\NormalTok{( }\AttributeTok{x =}\NormalTok{ dk.mcmc3.b}\SpecialCharTok{$}\NormalTok{k, }\AttributeTok{prob=}\ConstantTok{TRUE}\NormalTok{, }\AttributeTok{breaks=}\NormalTok{bks, }\AttributeTok{border=}\ConstantTok{NA}\NormalTok{,}
      \AttributeTok{col=}\FunctionTok{rgb}\NormalTok{(}\DecValTok{0}\NormalTok{, }\DecValTok{0}\NormalTok{, }\DecValTok{1}\NormalTok{, .}\DecValTok{5}\NormalTok{), }\AttributeTok{las=}\DecValTok{1}\NormalTok{, }\AttributeTok{xlab=}\StringTok{"kappa"}\NormalTok{,}
      \AttributeTok{xlim=}\FunctionTok{c}\NormalTok{(}\DecValTok{0}\NormalTok{,}\DecValTok{100}\NormalTok{), }\AttributeTok{ylim=}\FunctionTok{c}\NormalTok{(}\DecValTok{0}\NormalTok{,.}\DecValTok{055}\NormalTok{))}
\FunctionTok{hist}\NormalTok{( }\AttributeTok{x =}\NormalTok{ dk.mcmc3}\SpecialCharTok{$}\NormalTok{k, }\AttributeTok{prob=}\ConstantTok{TRUE}\NormalTok{, }\AttributeTok{breaks=}\NormalTok{bks, }\AttributeTok{border=}\ConstantTok{NA}\NormalTok{,}
      \AttributeTok{col=}\FunctionTok{rgb}\NormalTok{(.}\DecValTok{5}\NormalTok{, .}\DecValTok{5}\NormalTok{, .}\DecValTok{5}\NormalTok{, .}\DecValTok{5}\NormalTok{), }\AttributeTok{add=}\ConstantTok{TRUE}\NormalTok{)}
\end{Highlighting}
\end{Shaded}

\includegraphics{220303_Practical_session03_00_ex1_files/figure-latex/unnamed-chunk-24-1.pdf}

Now, as we did in the previous example, we will compute the mean and the
standard deviation for each chain. Then, we will plot the corresponding
densities so it is easier to see which chains are more or less
efficient:

\begin{Shaded}
\begin{Highlighting}[]
\CommentTok{\# A) Calculate the posterior means and s.d for each chain.}
\CommentTok{\# Compute means for efficient chains (they are quite similar)}
\FunctionTok{mean}\NormalTok{( dk.mcmc}\SpecialCharTok{$}\NormalTok{d ); }\FunctionTok{mean}\NormalTok{( dk.mcmc.b}\SpecialCharTok{$}\NormalTok{d )}
\end{Highlighting}
\end{Shaded}

\begin{verbatim}
## [1] 0.1042516
\end{verbatim}

\begin{verbatim}
## [1] 0.104695
\end{verbatim}

\begin{Shaded}
\begin{Highlighting}[]
\FunctionTok{mean}\NormalTok{( dk.mcmc}\SpecialCharTok{$}\NormalTok{k ); }\FunctionTok{mean}\NormalTok{( dk.mcmc.b}\SpecialCharTok{$}\NormalTok{k )}
\end{Highlighting}
\end{Shaded}

\begin{verbatim}
## [1] 29.12619
\end{verbatim}

\begin{verbatim}
## [1] 29.10531
\end{verbatim}

\begin{Shaded}
\begin{Highlighting}[]
\CommentTok{\# Compute means for inefficient chains (not so similar)}
\FunctionTok{mean}\NormalTok{( dk.mcmc3}\SpecialCharTok{$}\NormalTok{d ); }\FunctionTok{mean}\NormalTok{( dk.mcmc3.b}\SpecialCharTok{$}\NormalTok{d )}
\end{Highlighting}
\end{Shaded}

\begin{verbatim}
## [1] 0.1020017
\end{verbatim}

\begin{verbatim}
## [1] 0.1057856
\end{verbatim}

\begin{Shaded}
\begin{Highlighting}[]
\FunctionTok{mean}\NormalTok{( dk.mcmc3}\SpecialCharTok{$}\NormalTok{k ); }\FunctionTok{mean}\NormalTok{( dk.mcmc3.b}\SpecialCharTok{$}\NormalTok{k )}
\end{Highlighting}
\end{Shaded}

\begin{verbatim}
## [1] 27.00842
\end{verbatim}

\begin{verbatim}
## [1] 27.25209
\end{verbatim}

\begin{Shaded}
\begin{Highlighting}[]
\CommentTok{\# Standard error of the means for efficient chains}
\FunctionTok{sqrt}\NormalTok{( }\DecValTok{1}\SpecialCharTok{/}\FloatTok{1e4} \SpecialCharTok{*} \FunctionTok{var}\NormalTok{( dk.mcmc}\SpecialCharTok{$}\NormalTok{d ) }\SpecialCharTok{/} \FloatTok{0.23}\NormalTok{ ) }\CommentTok{\# roughly 2.5e{-}4}
\end{Highlighting}
\end{Shaded}

\begin{verbatim}
## [1] 0.0002406502
\end{verbatim}

\begin{Shaded}
\begin{Highlighting}[]
\FunctionTok{sqrt}\NormalTok{( }\DecValTok{1}\SpecialCharTok{/}\FloatTok{1e4} \SpecialCharTok{*} \FunctionTok{var}\NormalTok{( dk.mcmc}\SpecialCharTok{$}\NormalTok{k ) }\SpecialCharTok{/} \FloatTok{0.20}\NormalTok{ ) }\CommentTok{\# roughly 0.2}
\end{Highlighting}
\end{Shaded}

\begin{verbatim}
## [1] 0.2328288
\end{verbatim}

\begin{Shaded}
\begin{Highlighting}[]
\CommentTok{\# Standard error of the means for inefficient chain}
\FunctionTok{sqrt}\NormalTok{( }\DecValTok{1}\SpecialCharTok{/}\FloatTok{1e4} \SpecialCharTok{*} \FunctionTok{var}\NormalTok{( dk.mcmc3}\SpecialCharTok{$}\NormalTok{d ) }\SpecialCharTok{/} \FloatTok{0.015}\NormalTok{ ) }\CommentTok{\# roughly 9.7e{-}4}
\end{Highlighting}
\end{Shaded}

\begin{verbatim}
## [1] 0.0009067983
\end{verbatim}

\begin{Shaded}
\begin{Highlighting}[]
\FunctionTok{sqrt}\NormalTok{( }\DecValTok{1}\SpecialCharTok{/}\FloatTok{1e4} \SpecialCharTok{*} \FunctionTok{var}\NormalTok{( dk.mcmc3}\SpecialCharTok{$}\NormalTok{k ) }\SpecialCharTok{/} \FloatTok{0.003}\NormalTok{ ) }\CommentTok{\# roughly 1.6}
\end{Highlighting}
\end{Shaded}

\begin{verbatim}
## [1] 1.412789
\end{verbatim}

\begin{Shaded}
\begin{Highlighting}[]
\CommentTok{\# B) Plot densities (smoothed histograms) for the efficient and}
\CommentTok{\#    inefficient chains.}
\CommentTok{\# Set value to scale the kernel densities for the MCMCs}
\NormalTok{adj }\OtherTok{\textless{}{-}} \FloatTok{1.5}
\FunctionTok{par}\NormalTok{( }\AttributeTok{mfrow =} \FunctionTok{c}\NormalTok{( }\DecValTok{1}\NormalTok{,}\DecValTok{2}\NormalTok{ ) )}
\CommentTok{\# Efficient chains}
\FunctionTok{plot}\NormalTok{( }\AttributeTok{x =} \FunctionTok{density}\NormalTok{( }\AttributeTok{x =}\NormalTok{ dk.mcmc.b}\SpecialCharTok{$}\NormalTok{k, }\AttributeTok{adjust =}\NormalTok{ adj ), }\AttributeTok{col =} \StringTok{"blue"}\NormalTok{, }\AttributeTok{las =} \DecValTok{1}\NormalTok{,}
      \AttributeTok{xlim  =} \FunctionTok{c}\NormalTok{( }\DecValTok{0}\NormalTok{, }\DecValTok{100}\NormalTok{ ), }\AttributeTok{ylim =} \FunctionTok{c}\NormalTok{( }\DecValTok{0}\NormalTok{, .}\DecValTok{05}\NormalTok{ ), }\AttributeTok{xaxs =} \StringTok{"i"}\NormalTok{, }\AttributeTok{yaxs =} \StringTok{"i"}\NormalTok{ )}
\FunctionTok{lines}\NormalTok{( }\AttributeTok{x =} \FunctionTok{density}\NormalTok{( }\AttributeTok{x =}\NormalTok{ dk.mcmc}\SpecialCharTok{$}\NormalTok{k, }\AttributeTok{adjust =}\NormalTok{ adj ), }\AttributeTok{col =} \StringTok{"black"}\NormalTok{ )}
\CommentTok{\# Inefficient chains}
\FunctionTok{plot}\NormalTok{( }\AttributeTok{x =} \FunctionTok{density}\NormalTok{( dk.mcmc3.b}\SpecialCharTok{$}\NormalTok{k, }\AttributeTok{adjust =}\NormalTok{ adj ), }\AttributeTok{col =} \StringTok{"blue"}\NormalTok{, }\AttributeTok{las =} \DecValTok{1}\NormalTok{,}
      \AttributeTok{xlim =} \FunctionTok{c}\NormalTok{(}\DecValTok{0}\NormalTok{, }\DecValTok{100}\NormalTok{), }\AttributeTok{ylim =} \FunctionTok{c}\NormalTok{( }\DecValTok{0}\NormalTok{, .}\DecValTok{05}\NormalTok{ ), }\AttributeTok{xaxs =} \StringTok{"i"}\NormalTok{, }\AttributeTok{yaxs =} \StringTok{"i"}\NormalTok{ )}
\FunctionTok{lines}\NormalTok{( }\AttributeTok{x =} \FunctionTok{density}\NormalTok{( }\AttributeTok{x =}\NormalTok{ dk.mcmc3}\SpecialCharTok{$}\NormalTok{k, }\AttributeTok{adjust =}\NormalTok{ adj ), }\AttributeTok{col =} \StringTok{"black"}\NormalTok{ )}
\end{Highlighting}
\end{Shaded}

\includegraphics{220303_Practical_session03_00_ex1_files/figure-latex/unnamed-chunk-25-1.pdf}

\end{document}
