% Options for packages loaded elsewhere
\PassOptionsToPackage{unicode}{hyperref}
\PassOptionsToPackage{hyphens}{url}
%
\documentclass[
]{article}
\title{Practical session 2}
\author{Sandra Alvarez-Carretero}
\date{2022/03/01}

\usepackage{amsmath,amssymb}
\usepackage{lmodern}
\usepackage{iftex}
\ifPDFTeX
  \usepackage[T1]{fontenc}
  \usepackage[utf8]{inputenc}
  \usepackage{textcomp} % provide euro and other symbols
\else % if luatex or xetex
  \usepackage{unicode-math}
  \defaultfontfeatures{Scale=MatchLowercase}
  \defaultfontfeatures[\rmfamily]{Ligatures=TeX,Scale=1}
\fi
% Use upquote if available, for straight quotes in verbatim environments
\IfFileExists{upquote.sty}{\usepackage{upquote}}{}
\IfFileExists{microtype.sty}{% use microtype if available
  \usepackage[]{microtype}
  \UseMicrotypeSet[protrusion]{basicmath} % disable protrusion for tt fonts
}{}
\makeatletter
\@ifundefined{KOMAClassName}{% if non-KOMA class
  \IfFileExists{parskip.sty}{%
    \usepackage{parskip}
  }{% else
    \setlength{\parindent}{0pt}
    \setlength{\parskip}{6pt plus 2pt minus 1pt}}
}{% if KOMA class
  \KOMAoptions{parskip=half}}
\makeatother
\usepackage{xcolor}
\IfFileExists{xurl.sty}{\usepackage{xurl}}{} % add URL line breaks if available
\IfFileExists{bookmark.sty}{\usepackage{bookmark}}{\usepackage{hyperref}}
\hypersetup{
  pdftitle={Practical session 2},
  pdfauthor={Sandra Alvarez-Carretero},
  hidelinks,
  pdfcreator={LaTeX via pandoc}}
\urlstyle{same} % disable monospaced font for URLs
\usepackage[margin=1in]{geometry}
\usepackage{color}
\usepackage{fancyvrb}
\newcommand{\VerbBar}{|}
\newcommand{\VERB}{\Verb[commandchars=\\\{\}]}
\DefineVerbatimEnvironment{Highlighting}{Verbatim}{commandchars=\\\{\}}
% Add ',fontsize=\small' for more characters per line
\usepackage{framed}
\definecolor{shadecolor}{RGB}{248,248,248}
\newenvironment{Shaded}{\begin{snugshade}}{\end{snugshade}}
\newcommand{\AlertTok}[1]{\textcolor[rgb]{0.94,0.16,0.16}{#1}}
\newcommand{\AnnotationTok}[1]{\textcolor[rgb]{0.56,0.35,0.01}{\textbf{\textit{#1}}}}
\newcommand{\AttributeTok}[1]{\textcolor[rgb]{0.77,0.63,0.00}{#1}}
\newcommand{\BaseNTok}[1]{\textcolor[rgb]{0.00,0.00,0.81}{#1}}
\newcommand{\BuiltInTok}[1]{#1}
\newcommand{\CharTok}[1]{\textcolor[rgb]{0.31,0.60,0.02}{#1}}
\newcommand{\CommentTok}[1]{\textcolor[rgb]{0.56,0.35,0.01}{\textit{#1}}}
\newcommand{\CommentVarTok}[1]{\textcolor[rgb]{0.56,0.35,0.01}{\textbf{\textit{#1}}}}
\newcommand{\ConstantTok}[1]{\textcolor[rgb]{0.00,0.00,0.00}{#1}}
\newcommand{\ControlFlowTok}[1]{\textcolor[rgb]{0.13,0.29,0.53}{\textbf{#1}}}
\newcommand{\DataTypeTok}[1]{\textcolor[rgb]{0.13,0.29,0.53}{#1}}
\newcommand{\DecValTok}[1]{\textcolor[rgb]{0.00,0.00,0.81}{#1}}
\newcommand{\DocumentationTok}[1]{\textcolor[rgb]{0.56,0.35,0.01}{\textbf{\textit{#1}}}}
\newcommand{\ErrorTok}[1]{\textcolor[rgb]{0.64,0.00,0.00}{\textbf{#1}}}
\newcommand{\ExtensionTok}[1]{#1}
\newcommand{\FloatTok}[1]{\textcolor[rgb]{0.00,0.00,0.81}{#1}}
\newcommand{\FunctionTok}[1]{\textcolor[rgb]{0.00,0.00,0.00}{#1}}
\newcommand{\ImportTok}[1]{#1}
\newcommand{\InformationTok}[1]{\textcolor[rgb]{0.56,0.35,0.01}{\textbf{\textit{#1}}}}
\newcommand{\KeywordTok}[1]{\textcolor[rgb]{0.13,0.29,0.53}{\textbf{#1}}}
\newcommand{\NormalTok}[1]{#1}
\newcommand{\OperatorTok}[1]{\textcolor[rgb]{0.81,0.36,0.00}{\textbf{#1}}}
\newcommand{\OtherTok}[1]{\textcolor[rgb]{0.56,0.35,0.01}{#1}}
\newcommand{\PreprocessorTok}[1]{\textcolor[rgb]{0.56,0.35,0.01}{\textit{#1}}}
\newcommand{\RegionMarkerTok}[1]{#1}
\newcommand{\SpecialCharTok}[1]{\textcolor[rgb]{0.00,0.00,0.00}{#1}}
\newcommand{\SpecialStringTok}[1]{\textcolor[rgb]{0.31,0.60,0.02}{#1}}
\newcommand{\StringTok}[1]{\textcolor[rgb]{0.31,0.60,0.02}{#1}}
\newcommand{\VariableTok}[1]{\textcolor[rgb]{0.00,0.00,0.00}{#1}}
\newcommand{\VerbatimStringTok}[1]{\textcolor[rgb]{0.31,0.60,0.02}{#1}}
\newcommand{\WarningTok}[1]{\textcolor[rgb]{0.56,0.35,0.01}{\textbf{\textit{#1}}}}
\usepackage{graphicx}
\makeatletter
\def\maxwidth{\ifdim\Gin@nat@width>\linewidth\linewidth\else\Gin@nat@width\fi}
\def\maxheight{\ifdim\Gin@nat@height>\textheight\textheight\else\Gin@nat@height\fi}
\makeatother
% Scale images if necessary, so that they will not overflow the page
% margins by default, and it is still possible to overwrite the defaults
% using explicit options in \includegraphics[width, height, ...]{}
\setkeys{Gin}{width=\maxwidth,height=\maxheight,keepaspectratio}
% Set default figure placement to htbp
\makeatletter
\def\fps@figure{htbp}
\makeatother
\setlength{\emergencystretch}{3em} % prevent overfull lines
\providecommand{\tightlist}{%
  \setlength{\itemsep}{0pt}\setlength{\parskip}{0pt}}
\setcounter{secnumdepth}{-\maxdimen} % remove section numbering
\ifLuaTeX
  \usepackage{selnolig}  % disable illegal ligatures
\fi

\begin{document}
\maketitle

\hypertarget{setting-your-working-environment}{%
\section{Setting your working
environment}\label{setting-your-working-environment}}

First, we will clean and set our working environment. It is important
that, whenever you start working on a project, you clean your working
environment to avoid issues with objects generated in previous
sessions/projects. If you want to do this using the command line, you
can do the following:

\begin{Shaded}
\begin{Highlighting}[]
\CommentTok{\# Clean environment }
\FunctionTok{rm}\NormalTok{( }\AttributeTok{list =} \FunctionTok{ls}\NormalTok{( ) )}

\CommentTok{\# Set working directory with package \textasciigrave{}rstudioapi\textasciigrave{}:}
\CommentTok{\#}
\CommentTok{\# 1. Load the package \textasciigrave{}rstudioapi\textasciigrave{}. If you do not have }
\CommentTok{\#    it installed, then uncomment and run the}
\CommentTok{\#    command below}
\CommentTok{\# install.packages( "rstudioapi" )}
\FunctionTok{library}\NormalTok{( rstudioapi ) }
\CommentTok{\# 2. Get the path to current open R script}
\NormalTok{path\_to\_file }\OtherTok{\textless{}{-}} \FunctionTok{getActiveDocumentContext}\NormalTok{()}\SpecialCharTok{$}\NormalTok{path}
\NormalTok{wd           }\OtherTok{\textless{}{-}} \FunctionTok{paste}\NormalTok{( }\FunctionTok{dirname}\NormalTok{( path\_to\_file ), }\StringTok{"/"}\NormalTok{, }\AttributeTok{sep =} \StringTok{""}\NormalTok{ )}
\FunctionTok{setwd}\NormalTok{( wd )}
\end{Highlighting}
\end{Shaded}

Now, any data that you generate when you run the commands in this R
script will be saved in the directory you have defined above (unless you
specify otherwise when saving the data!).

\begin{center}\rule{0.5\linewidth}{0.5pt}\end{center}

\hypertarget{combining-the-likelihood-and-the-prior}{%
\section{Combining the likelihood and the
prior}\label{combining-the-likelihood-and-the-prior}}

\hypertarget{exercise-1}{%
\subsection{EXERCISE 1}\label{exercise-1}}

\emph{\textbf{SCENARIO}} We have started our field work and we are
studying a specific tree species. We want to take a sample from the
different trees planted in the area, so we collecte one leave per tree
and measure the width (cm). The results are the following:

\begin{Shaded}
\begin{Highlighting}[]
\CommentTok{\# Save collected measurements as data, D}
\NormalTok{D }\OtherTok{\textless{}{-}} \FunctionTok{c}\NormalTok{( }\FloatTok{0.5}\NormalTok{, }\FloatTok{0.8}\NormalTok{, }\FloatTok{0.9}\NormalTok{, }\FloatTok{1.0}\NormalTok{, }\FloatTok{1.15}\NormalTok{, }\FloatTok{0.9}\NormalTok{, }\FloatTok{0.8}\NormalTok{, }\FloatTok{0.9}\NormalTok{, }\FloatTok{1.7}\NormalTok{, }\DecValTok{2}\NormalTok{, }\FloatTok{1.1}\NormalTok{ )}
\CommentTok{\# Count number of observations in D}
\NormalTok{n }\OtherTok{\textless{}{-}} \FunctionTok{length}\NormalTok{( D )}
\CommentTok{\# Estimate variance and sd }
\FunctionTok{sd}\NormalTok{( D ) }\CommentTok{\# 0.43}
\end{Highlighting}
\end{Shaded}

\begin{verbatim}
## [1] 0.4279443
\end{verbatim}

\begin{Shaded}
\begin{Highlighting}[]
\CommentTok{\# Estimate mean }
\FunctionTok{mean}\NormalTok{( D ) }\CommentTok{\# 1.07}
\end{Highlighting}
\end{Shaded}

\begin{verbatim}
## [1] 1.068182
\end{verbatim}

\begin{Shaded}
\begin{Highlighting}[]
\CommentTok{\# Plot data (use \textasciigrave{}breaks\textasciigrave{} so we can better visualise }
\CommentTok{\# our data)}
\FunctionTok{hist}\NormalTok{( D, }\AttributeTok{breaks=}\DecValTok{6}\NormalTok{ )}
\end{Highlighting}
\end{Shaded}

\includegraphics{220301_ExampleLecture_WRONGCODE_ignore_files/figure-latex/unnamed-chunk-2-1.pdf}
By looking at our data, we might consider that the best probability
distribution that we can use to model my likelihood is the normal
distribution. This function would help us understand the probability of
observing my data, \(D\), given the values that we randomly choose for
our parameters of interest. For this example, we will assume the
estimates we get from our collected data so, if we put those in a
formula:

\(lnL(D|\theta)=lnL(D|\mu,sd)=lnL(D\mu=1.07, sd=0.43)\)

\begin{Shaded}
\begin{Highlighting}[]
\CommentTok{\# L\_norm \textless{}{-} function( mu = 1.07, sd = 0.43, D ) \{}
\CommentTok{\#L\_norm \textless{}{-} function( mu, sd, D ) \{}
\NormalTok{L\_norm }\OtherTok{\textless{}{-}} \ControlFlowTok{function}\NormalTok{( D, mu, sd ) \{}
  \CommentTok{\# Define the mean and the standard deviation, which are passed }
  \CommentTok{\# to one argument as a numeric vector of length 2}
\NormalTok{  mu    }\OtherTok{\textless{}{-}}\NormalTok{ mu}
\NormalTok{  sigma }\OtherTok{\textless{}{-}}\NormalTok{ sd}
  \CommentTok{\# Define log{-}normal distribution, a sum of all the densities }
  \CommentTok{\# calculated for each data point that we have in our dataset, D}
\NormalTok{  L   }\OtherTok{\textless{}{-}} \FunctionTok{sum}\NormalTok{( }\FunctionTok{dnorm}\NormalTok{( }\AttributeTok{x =}\NormalTok{ D, }\AttributeTok{mean =}\NormalTok{ mu, }\AttributeTok{sd =}\NormalTok{ sigma ) )}
  \CommentTok{\# Return likelihood }
  \FunctionTok{return}\NormalTok{( L )}
\NormalTok{\}}
\end{Highlighting}
\end{Shaded}

Even though we know the likelihood, we have not yet incorporated any
information about our parameters of interest before: we had not used any
prior distribution.

After contacting some botanists working in the field work we have
collected our data, we know that for the past 10 years they have seen
that the average (in cm) for the leaves is around \(0.85\). With this
information, we would be able to build a prior.

\emph{\textbf{QUESTION}} \textbf{1. Which probability distribution do
you think would be more sensible to use as a prior for \(\mu\), our
parameter of interest. In other words, what probability distribution
would you use to represent the prior on the mean width of the leaves?}

\begin{quote}
\emph{\textbf{ANSWER}}
\end{quote}

A normal distribution given that our collaborators have given us some
data about the mean width of the leaves and that our likelihood is also
a normal distribution.

\emph{\textbf{QUESTION}} \textbf{2. Which would be the values for
\(\mu\) and the standard deviation that you would use to define your
prior? What do you base this selection?}

\begin{quote}
\emph{\textbf{ANSWER}}
\end{quote}

There is not a unique answer. For instance, we could assume that
\(parameter\pm 2\times sd > 0\) given that our prior distribution will
be a normal distribution. Therefore:

\(\mu_{leave} - 2\times sd{leave} > 0\)
\(\frac{\mu_{leave}}{2} > sd{levae}\) \(\frac{0.85}{2} > sd{leave}\)
\(sd{leave} = 0.425\)

Therefore, \(\mu_{leave} \~ N(0.85,0.425^2)\)

\begin{center}\rule{0.5\linewidth}{0.5pt}\end{center}

\hypertarget{posterior-unnormalised-posterior-and-marginal-likelihood}{%
\section{Posterior, unnormalised posterior, and marginal
likelihood}\label{posterior-unnormalised-posterior-and-marginal-likelihood}}

\textbf{How would you analytically compute the posterior distribution,
i.e., \(P(\theta|D)=P(\mu|D)\)?}

In the previous section, we have seen how we can fit normal
distributions for the likelihood and the prior (i.e., the numerator).
Nevertheless, we have not yet talked about the denominator, the marginal
likelihood:

\(posterior = \frac{prior\times likelihood}{marginal\_likelihood}\)

Normally, we would want to avoid computing the marginal likelihood
because it tends to be overly complicated. In this case, however, the
marginal likelihood is not too complicated and can is computationally
tractable. We defined the likelihood function above, so now we can
define the prior:

\begin{Shaded}
\begin{Highlighting}[]
\CommentTok{\# Define the prior as a normal distribution }
\CommentTok{\#prior\_fun \textless{}{-} function( mu\_l = 0.85, sd\_l = 0.425, D ) \{}
\CommentTok{\#prior\_fun \textless{}{-} function( mu\_l, sd\_l, D ) \{}
\NormalTok{prior\_fun }\OtherTok{\textless{}{-}} \ControlFlowTok{function}\NormalTok{( D, mu\_l, sd\_l ) \{}
  \CommentTok{\# Define the mean and the standard deviation, which are passed }
  \CommentTok{\# to one argument as a numeric vector of length 2}
\NormalTok{  mu\_2    }\OtherTok{\textless{}{-}}\NormalTok{ mu\_l}
\NormalTok{  sigma\_2 }\OtherTok{\textless{}{-}}\NormalTok{ sd\_l}
  \CommentTok{\# Define log{-}normal distribution, a sum of all the densities }
  \CommentTok{\# calculated for each data point that we have in our dataset, D}
\NormalTok{  L   }\OtherTok{\textless{}{-}} \FunctionTok{sum}\NormalTok{( }\FunctionTok{dnorm}\NormalTok{( }\AttributeTok{x =}\NormalTok{ D, }\AttributeTok{mean =}\NormalTok{ mu\_2, }\AttributeTok{sd =}\NormalTok{ sigma\_2 ) )}
  \CommentTok{\# Return likelihood }
  \FunctionTok{return}\NormalTok{( L )}
\NormalTok{\}}
\end{Highlighting}
\end{Shaded}

The next step is to compute the numerator, what we call the
\textbf{unnormalised posterior} (more on this when we learn about
MCMC!). The only thing we need to do is multiplicate the prior by the
likelihood. We can do this because we have already defined the
corresponding functions:

\begin{Shaded}
\begin{Highlighting}[]
\CommentTok{\# Defined unnormalised posterior}
\CommentTok{\# unnorm\_post \textless{}{-} function( mu = 1.07, sd = 0.43, mu\_l = 0.85, sd\_l= 0.425, D )\{}
\CommentTok{\#unnorm\_post \textless{}{-} function( mu, sd, mu\_l, sd\_l, D )\{}
\NormalTok{unnorm\_post }\OtherTok{\textless{}{-}} \ControlFlowTok{function}\NormalTok{( D, mu, sd, mu\_l, sd\_l )\{}
  \CommentTok{\# posterior \textasciitilde{} prior x lnL}
  \FunctionTok{return}\NormalTok{( }\FunctionTok{prior\_fun}\NormalTok{( }\AttributeTok{mu\_l =}\NormalTok{ mu\_l, }\AttributeTok{sd\_l =}\NormalTok{ sd\_l, }\AttributeTok{D =}\NormalTok{ D )}\SpecialCharTok{*}\FunctionTok{L\_norm}\NormalTok{( }\AttributeTok{mu =}\NormalTok{ mu, }\AttributeTok{sd =}\NormalTok{ sd, }\AttributeTok{D =}\NormalTok{ D ) )}
\NormalTok{\}}
\end{Highlighting}
\end{Shaded}

So far, we have defined three functions for three probability
distributions: the prior, the likelihood, and the unnormalised
posterior. So now\ldots{} how de we manage to compute \textbf{the
posterior}? Well, we will need to incorporate the marginal likelihood,
which basically is a constant that needs to integrate to 1!

Remember that this term is the probability of the data, of everything we
can observe. In this way, it is an integral over the joint probability:

\(P(marginal\_likelihood)=\int{prior\times lnL}\)

To calculate this integral, we could use the function
\texttt{integrate}, which will target the unnormalised posterior:

\begin{Shaded}
\begin{Highlighting}[]
\CommentTok{\# }\AlertTok{NOTE}\CommentTok{: \textasciigrave{}f\textasciigrave{} should be an R function taking a numeric first argument and returning a numeric vector of the same length. Vectorize will make \textasciigrave{}f\textasciigrave{} a such function that returns the same length output as input.}
\CommentTok{\# Source: https://stackoverflow.com/questions/43818574/error{-}in{-}integrate{-}evaluation{-}of{-}function{-}gave{-}a{-}result{-}of{-}wrong{-}length}
\FunctionTok{integrate}\NormalTok{( }\AttributeTok{f=}\FunctionTok{Vectorize}\NormalTok{(unnorm\_post), }\AttributeTok{lower=}\DecValTok{0}\NormalTok{, }\AttributeTok{upper=}\ConstantTok{Inf}\NormalTok{, }\AttributeTok{abs.tol =} \DecValTok{0}\NormalTok{, }
           \AttributeTok{mu =} \FloatTok{1.07}\NormalTok{, }\AttributeTok{sd =} \FloatTok{0.43}\NormalTok{, }\AttributeTok{mu\_l =} \FloatTok{0.85}\NormalTok{, }\AttributeTok{sd\_l=} \FloatTok{0.425}\NormalTok{ )}
\end{Highlighting}
\end{Shaded}

\begin{verbatim}
## 0.6171196 with absolute error < 1.5e-05
\end{verbatim}

\begin{Shaded}
\begin{Highlighting}[]
\CommentTok{\# 0.6171196 with absolute error \textless{} 1.5e{-}05}
\CommentTok{\# The marginal likelihood, this constant, will be: }
\NormalTok{C  }\OtherTok{\textless{}{-}} \DecValTok{1} \SpecialCharTok{/} \FloatTok{0.6171196}
\end{Highlighting}
\end{Shaded}

If now I want to compute my normalised posterior distribution:

\begin{Shaded}
\begin{Highlighting}[]
\CommentTok{\# posterior = C * prior * L = C * unnormalised\_posterior}
\NormalTok{C  }\OtherTok{\textless{}{-}} \DecValTok{1} \SpecialCharTok{/} \FloatTok{0.6171196}
\NormalTok{post\_fun }\OtherTok{\textless{}{-}} \ControlFlowTok{function}\NormalTok{( D, C, mu, sd, mu\_l, sd\_l )\{}
  \FunctionTok{return}\NormalTok{( C }\SpecialCharTok{*} \FunctionTok{unnorm\_post}\NormalTok{( D, mu, sd, mu\_l, sd\_l ) )}
\NormalTok{\}}

\CommentTok{\# Compute posterior}
\FunctionTok{post\_fun}\NormalTok{( }\AttributeTok{D =}\NormalTok{ D, }\AttributeTok{C =}\NormalTok{ C, }\AttributeTok{mu =} \FloatTok{1.07}\NormalTok{, }\AttributeTok{sd =} \FloatTok{0.43}\NormalTok{, }\AttributeTok{mu\_l =} \FloatTok{0.85}\NormalTok{, }\AttributeTok{sd\_l=} \FloatTok{0.425}\NormalTok{ ) }\CommentTok{\# 97.64066}
\end{Highlighting}
\end{Shaded}

\begin{verbatim}
## [1] 97.64066
\end{verbatim}

\end{document}
