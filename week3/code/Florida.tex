\documentclass[10pt]{article}
\usepackage[utf8]{inputenc}
\usepackage{graphicx}
\usepackage{a4wide}
\usepackage{geometry}
\addtolength{\topmargin}{-1.5in}
\addtolength{\oddsidemargin}{-2in}
\addtolength{\evensidemargin}{-2in}
\newgeometry{top = 0cm, bottom = 1cm}

\title{Is Florida getting warmer?}
\author{Eamonn Murphy | October 2021}
\date{}

\begin{document}

\maketitle

\thispagestyle{empty}

\section{Introduction}
There is widespread scientific consensus that global warming is a threat to biodiversity. Key West in Florida is close to many unique habitats, such as the Everglades National Park \cite{junk_brown_campbell_finlayson_gopal_ramberg_warner_2006}. This analysis looks at the temperature data from the 20th century in Key West, to assess the evidence of warming there over this period.

\section{Methods}
A script was created in R to analyse the dataset. Spearman's correlation 
coefficient ($\rho$) was calculated between the years and average temperatures. $\rho$ was used as it checks for a monotonic, or rank order, relationship, which does not need to be linear \cite{dewinter_gosling_potter_2016}. In  order to obtain a p-value, permutation testing was used. The order of temperatures was shuffled randomly between years 10,000  times, and $\rho$ calculated for each random order. The p-value would thus be the number of random correlations greater than the base correlation, divided by 10,000.

\section{Results}
\begin{center}
    \includegraphics[scale = 0.4]{../results/temp_year_scatter.png}
    \includegraphics[scale = 0.4]{../results/coeff_distro.png}

    Figure One: (A) Temperature (C) vs. Year for Florida dataset. Line of best fit
    added from linear model. (B) Distribution of $\rho$
    for randomly permuted temperature orders.
\end{center}

$\rho$ for the relationship between temperature and year was 0.526.
10,000 permutation tests were run, randomly shuffling the years. $\rho$ was calculated for each of these random orders. The
distribution of these is displayed in Figure One (B). None of these correlations 
were greater than the originally calculated correlation, meaning that the p-value 
is less than 1/10000 (<0.0001). This indicates that the observed increase in 
temperature over the century of data is very likely to be a true increase, not 
due to random chance. These data indicate the potential impact of further global warming on ecosystems in Florida.

\bibliographystyle{apalike}
\bibliography{Florida}

\end{document}