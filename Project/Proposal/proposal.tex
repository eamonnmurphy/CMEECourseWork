\documentclass[11pt,titlepage]{article}
\usepackage{setspace}
\usepackage[utf8]{inputenc}
\usepackage[a4paper, margin=2cm]{geometry}
\usepackage{caption}
\usepackage{subcaption}
\usepackage{lineno}
\onehalfspacing

\title{Proposal for the MSc Computational Methods in Ecology Evolution Thesis}
\author{Eamonn Murphy}
\date{April 2022}

\begin{document}
	
	\maketitle
	
	\linenumbers
	\section{Keywords}
	
	Phylogenetics, genetics, conservation, extinction, restoration
	
	\section{Introduction}
	Many approaches are being used to tackle the current mass extinction event, which is likely to continue to get worse over the coming century. Made possible by recent advances in biotechnology, a new potential method has been put forward: restoration of extinct species using stored genetic data. An important part of this project would be prioritisation of the sample collection, as it is important to start this process soon for vulnerable species, to preserve as much genetic diversity as possible.
	
	Future biotechnological approaches to restore extinct species will rely on two biological factors: stored genetic data for the species of interest, and the survival of a closely related species to provide biological material, act as a surrogate etc. Thus, prioritisation approaches must address two key items. The species must be likely to go extinct in the relatively near future, and it must be likely that a closely related species survives. This differs from phylogenetic approaches for conservation prioritisation, where usually the idea is to preserve as much phylogenetic diversity as possible, making monophyletic species more important.
	\section{Proposed Methods}
	It is proposed to adapt phylogenetic approaches for conservation prioritisation to the particular constraints of species restoration. The mammalian clade will be the first priority, and methods will be developed here using R to come up with a prioritisation score or ranking for these species. Considerations will include how closely related is good enough for the species restoration to be viable, and what mathematical methods are best to use to calculate the prioritisation scores.
	\section{Anticipated Outputs and Outcomes}
	It is anticipated to have by the end of the project a usable prioritisation list for genetic biobanking aimed at future species restoration, within the mammalian clade. The methods and code used should also be available and ready to be applied to other clades, once adaptations have been made for considerations such as time of divergence between species.

	\section{Project Timeline}
	
	\section{Budget}
	\begin{table}
		\begin{center}
			\caption{Tabulated Budget Requests}
			\begin{tabular}{|c|c|c|c|}
				\hline
				Item & Cost per Unit & Number & Total Cost \\
				\hline
				Visit to ZSL and Edge & £25 & 5 & £125 \\
				\hline
				Reading Mathematical Models in Ecology and Evolution Conference & £10 & 3 & £30 \\
				\hline
				\textbf{Total}& & & £155 \\
				\hline
			\end{tabular}
		\end{center}
	\end{table}
	
	\newpage
	\section{Supervisor Signature}
	
	Name: James Rosindell \\
	Signature: \\
	Date: \\
	
	%\bibliography{}
	
\end{document}